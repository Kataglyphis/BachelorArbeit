\subsection{Eigenschaften}

Wie in \cite{Pet17} vorgestellt, macht man sich die Eigenschaften einer
blue noise Textur zu Nutze. Dabei werden im Folgenden, die dort bereit 
gestellten blue noise verteilten Texturen verwendet, welche anhand des in
\cite{ulichney1993void} vorgestellten Algorithmus erstellt wurden.
Die korrespondierenden Spektren werden mit Hilfe von \cite{FFTProgWeb} erstellt.

\subsubsection{Uniformität}
Die Uniformität(lat. \textit{uniformitas}-Einförmigkeit) garantiert uns 
wie in \cite{3288} eine gleichverteilte Wahrscheinlichkeitsdichtefunktion
mit zugehöriger gleichverteilter Wahrscheinlichkeitsfunktion. In \cite{Pet17}
sieht sie wie folgt aus: 

\begin{equation}\label{eq:uniformität}
    P(n \leq p) = p
\end{equation}

\subsubsection{Niedrige Frequenzen}
Niedrige Frequenzen sind in einer blue noise sehr wenig bis gar nicht 
vertreten. Dies ist an dem schwarzen Ring innerhalb der Fouriertransformierten
zu erkennen\ref{pic:blueNoiseFFT}.

\begin{figure}[H]\label{pic:blueNoiseFFT}
    \centering
    \begin{minipage}[t]{0.45\linewidth}
        \centering
        \includegraphics[width=\linewidth]{content/BlueNoise/Bilder/LDR_LLL1_0.png}
        \caption{$512^{2}$ blue noise Textur}
    \end{minipage}
    \hfill
    \begin{minipage}[t]{0.45\linewidth}
        \centering
        \includegraphics[width=\linewidth]{content/BlueNoise/Bilder/FFT_LDR_LLL1_0.png}
        \caption{Fourier Spektrum $512^{2}$ blue noise Textur}
    \end{minipage}
\end{figure}

\begin{figure}[H]\label{pic:bayerPatternFFT}
    \centering
    \begin{minipage}[t]{0.45\linewidth}
        \centering
        \includegraphics[width=\linewidth]{content/BlueNoise/Bilder/BayerMatrix.png}
        \caption{$512^{2}$ bayer pattern Textur}
    \end{minipage}
    \hfill
    \begin{minipage}[t]{0.45\linewidth}
        \centering
        \includegraphics[width=\linewidth]{content/BlueNoise/Bilder/FFT_BayerMatrix.png}
        \caption{Fourier Spektrum $512^{2}$ bayer pattern Textur}
    \end{minipage}
\end{figure}

\begin{figure}[H]\label{pic:tiledBlueNoiseFFT}
    \centering
    \begin{minipage}[t]{0.45\linewidth}
        \centering
        \includegraphics[width=\linewidth]{content/BlueNoise/Bilder/BlueNoise64Tiled.png}
        \caption{$512^{2}$ bayer pattern Textur}
    \end{minipage}
    \hfill
    \begin{minipage}[t]{0.45\linewidth}
        \centering
        \includegraphics[width=\linewidth]{content/BlueNoise/Bilder/FFT_BlueNoise64Tiled.png}
        \caption{Fourier Spektrum $512^{2}$ bayer pattern Textur}
    \end{minipage}
\end{figure}

\subsubsection{Isotropie}
Die Isotropie(altgr. \textit{isos}-gleich und \textit{tropos}-Richtung)
einer blue noise Textur wird ausgenutzt. Dabei haben wir in allen
Dimensionen (in dieser Arbeit werden Texturen mit zwei benutzt) 
die Unabhängigkeit einer Eigenschaft. 


\subsubsection{Kachelung}
Eine weitere nützliche Eigenschaft der blue noise Verteilung ist die 
Möglichkeit der Kachelung. 