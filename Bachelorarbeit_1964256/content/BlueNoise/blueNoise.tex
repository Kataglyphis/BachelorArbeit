\subsection{Eigenschaften}

Wie in \cite{Pet17} vorgestellt, macht man sich die Eigenschaften einer
blue noise Textur zu Nutze.
\subsubsection{Uniformität}
Die Uniformität(spätlat. \textit{uniformitas}-Einförmigkeit) 

\subsubsection{Niedrige Frequenzen}
Niedrige Frequenzen sind in einer blue noise sehr wenig bis gar nicht 
vertreten.

\subsubsection{Isotropie}
Die Isotropie(altgr. \textit{isos}-gleich und \textit{tropos}-Richtung)
einer blue noise Textur wird ausgenutzt. Dabei haben wir in allen
Dimensionen (in dieser Arbeit werden Texturen mit zwei benutzt) 
die Unabhängigkeit einer Eigenschaft. 

\subsubsection{Kachelung}
Eine weitere nützliche Eigenschaft der blue noise Verteilung ist die 
Möglichkeit der Kachelung. 