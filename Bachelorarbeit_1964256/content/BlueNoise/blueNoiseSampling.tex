%Betrachten wir Verfahren wie das \nameref{sec:quasi monte carlo} Verfahren, 
\label{subsec:dither sampling}
Nachdem wir die Eigenschaften der \nameref{ch:Content1:sec:blue noise} kennengelernt haben
können wir zusammen mit dem Verständnis über den \nameref{ch:Content1:sec:Path Tracer} und der 
zugrundeliegenden Monte-Carlo Integration \ref{eq:Monte-Carlo} das dithered sampling verstehen.
\textit{Dithering} ist das bewusste Einbringen eines Rauschens um entstehende Quantisierungsfehler
zu randomisieren \cite{georgiev2016blue}.
Klassischerweise wird eine zweidimensionale 
\nameref{ch:Content1:sec:blue noise} Textur verwendet um mit einer darauf aufbauenden Schwellenwertbildung
dieses Rauschen in ein Bild zu bringen. 
Hier wollen wir durch Dithering die Verteilung des 
entstehenden Monte-Carlo Integrationsfehlers verändern.\par
Grundlage des d-dimensionalen \nameref{ch:Content1:sec:Path Tracer} werden sowohl eine 
\nameref{ch:Content1:sec:blue noise}-Verteilung als auch Anfangswerte $s_{n}$ gleicher Dimension.
Damit konkretisiert sich die Monte-Carlo Integration \ref{eq:Monte-Carlo} mit Integrand f zu folgender 
Gleichung:
\begin{equation}\label{eq:concreteMonteCarlo}
    \frac{1}{N}\sum_{n=0}^{N-1}f(s_{n})
\end{equation}

\cite{georgiev2016blue} hat sich mit einer direkten Korrelation der Anfangswerte anhand einer 
\nameref{ch:Content1:sec:blue noise} Textur beschäftigt, in der Hoffnung das der Integrand f diese
Verteilung behält. \cite{hal02158423} zeigt jedoch die Limitierung dieser Methode auf und 
formuliert eine \nameref{ch:Content2:sec:a Posteriori}-Methode, welche wir uns für den 
Temporalen Algorithmus \ref{ch:Temporaler Algorithmus} zu Nutze machen werden.



