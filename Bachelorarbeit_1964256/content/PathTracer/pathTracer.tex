\subsubsection{Funktionsweise}
Bei der Pfadverfolgung
    \cite{kajiya1986rendering}
    \begin{equation}
        L(x,{x}^{'}) = g(x,{x}^{'}) * (L_{e}(x,{x}^{'}) + 
                        \int_{S}^{} b(x,{x}^{'},{x}^{''})
                        L({x}^{'},{x}^{''}d{x}^{''})) 
    \end{equation}
    Sie beschreibt den Energietransport \textit{L(x,${x}^{'})$} von einem Punkt ${x}^{'}$
    zu einem Punkt x. Dabei ist ein maßgebender Faktor, die relative Lage der beiden Punkte
    zueinander im Raum $g(x,{x}^{'})$. Ein weiterer Faktor ist die Abstrahlung 
    $L_{e}(x,{x}^{'})$ von ${x}^{'}$ nach x. Beinflusst wird der Energiefluss auch durch
    die bidirektionale Verteilungsfunktion $b(x,{x}^{'},{x}^{''})$, welche Aufschluss über
    das einfallende Licht von einem Punkt ${x}^{''}$ über ${x}^{'}$ zu x.

\subsubsection{Monte-Carlo-Integration}
    Mit der Monte Carlo Integration approximieren wir die Rendergleichung.
    \cite{KK02}
    \label{pic:MonteCarloIntegration}
    \begin{equation}
    \int_{{[0,1)}^s} f(x) dx \simeq \frac{1}{N}*\sum_{i=0}^{N-1}f(\xi_i)
    \end{equation}
    %%$\int_{{[0,1)}^s}f(x)dx$\simeq$\frac{1}{N}*\sum{i=0}^{N-1}f(\xi_i)$


