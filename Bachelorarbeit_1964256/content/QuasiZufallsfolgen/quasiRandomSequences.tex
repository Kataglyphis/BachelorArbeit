\subsection{Einleitung}

Um ungewollten Artefakten entgegenzuwirken wird empfohlen (so auch in \cite{bluenoisechrisschied})
mehrere verschiedene \nameref{ch:Content1:sec:blue noise} Texturen in ein Array zu laden 
und diese abwechselnd, randomisiert zu verwenden. Wir führen hier Kenntnisse über  
Quasi-Zufallsfolgen ein \cite{quasirandomsequencesbyRoberts} um aus den zuvor (nach Abbildung \ref{pic:tiled blue noise})
genannten Gründen nur eine kleine \nameref{ch:Content1:sec:blue noise} Textur verwenden
zu können. \par
Quasi-zufällige Sequenzen mit niedriger Abweichung sind deterministisch 
erzeugte Sequenzen, welche die Likelihood-Funktion der Clusterbildung

\begin{equation}\label{eq:Likeli-Hood-Gleichung}
    L_{x}(\delta) = f_{\delta}(x)
\end{equation}

minimieren (siehe Abschnitt \ref{ch:Content1:sec:blue noise:Niedrige Frequenzen}) und dabei die 
(siehe Abschnitt \nameref{ch:Content1:sec:blue noise:Uniformität}) erhalten. Beide Eigenschaften 
haben wir bereits bei dem \nameref{ch:Content1:sec:blue noise} Abschnitt behandelt.
Auf Quasi-zufällige Zahlenfolgen angewandt heißt das: Wir benutzen den gesamten Raum an Zufallszahlen
\textit{uniform} und hohe Frequenzen vermeiden Regionen, aus den viele Punkte kommen, die wieder 
punktarme Regionen zur Folge hätten.   
Im Folgenden wird für uns das Ziel sein, den quasi-zufälligen Zugriff auf eine Textur 
mit \nameref{subsec:2-Dimensionen} zu verstehen und nähern uns dabei über den Fall der 
\nameref{subsec:1-Dimension}.
Ein Weg quasi-Zufallsfolgen zu beschreiben sind die zugrundeliegenden Parameter. 
Wir werden uns hier Folgen anschauen, die als Basisparameter den 
goldenen Schnitt \ref{eq:GoldenerSchnitt} verwenden.

\subsection{Goldener Schnitt}

Der goldene Schnitt und die Generalisierung zur plastischen Zahl \ref{eq:plastische Zahl}
ist mitsamt ihren Eigenschaften bereits früh beschrieben worden \cite{vanderlaanplasticnumber}.
Als wichtiges Seitenverhältnis in der Architektur konnte Sie durch verschiedene Arbeiten ihren 
Eingang in die Mathematik finden \cite{krcadinac2006new}.

\begin{equation}\label{eq:GoldenerSchnitt}
    \Phi_{1} = \frac{1 + \sqrt{5}}{2} \approx 1.6180339887
\end{equation}

\label{subsec:1-Dimension}
\subsection{1-Dimension}
Wir benutzen Rekurrenz Sequenzen, basierend auf irrationalem 
Bruchrechnen der Form

\begin{equation}\label{eq:Rekurrenz Sequenz}
    R_{1}(\alpha) : t_n = s_0 + n\alpha(mod 1); n = 1,2,3,...
\end{equation}

wobei $\alpha \in \mathbb{I}$ und das (mod 1) einen \textit{"toroidally shift"}
bezeichnet. Will man mit dieser Formel eine Sequenz mit möglichst geringer
Abweichung schaffen so wählen wir den goldenen Schnitt \ref{eq:GoldenerSchnitt}
$\alpha = \Phi_{1}$ (siehe \cite{quasirandomsequencesbyRoberts}).

%\begin{algorithm}\label{alg:GoldenRatioSeq}
    %\State \alpha = (\frac{1}{\Phi_{2}},\frac{1}{\Phi_{2}^2})
    %\State t_{n} = {n*\alpha}, n = 1,2,3,...
%\end{algorithm}

\label{subsec:2-Dimensionen}
\subsection{2-Dimensionen}

Da uns die Generalisierung des goldenen Schnittes auf die Lösung der Gleichung
\ref{eq:GeneralisierungGoldenerSchnitt} führt

\begin{equation}\label{eq:GeneralisierungGoldenerSchnitt}
    x^{d+1} = x+1
\end{equation}

ist das Lösen der kubischen Gleichung \ref{eq:kubisch}

\begin{equation}\label{eq:kubisch}
    x^{3} - x - 1 = 0
\end{equation}

für den zweidimensionalen Fall nötig. Die Generalisierung und Erweiterung des goldenen 
Schnittes wurde bereits ausgiebig erforscht \cite{krcadinac2006new}.

Die sogenannte Plastische Zahl in ist die Lösung der
Gleichung \ref{eq:kubisch}

\begin{equation}\label{eq:plastische Zahl}
    \Phi_{2} \approx 1.32471795724
\end{equation}

Die eindimensionale Rekurrenzsequenz \ref{eq:Rekurrenz Sequenz} ist einfach erweiterbar 
für höhere Dimensionen.
\begin{equation}\label{eq:1 zu N - Dimensional}
    t_{n} = n\alpha(mod 1), n = 1,2,3,..
    \alpha = (\frac{1}{\Phi_{d}}, \frac{1}{\Phi_{d}^{2}}),
\end{equation}

Für den Texturzugriff in unserem Shader bei dem temporalen Algorithmus \ref{ch:Temporaler Algorithmus}
werden wir also wie folgt vorgehen:

\begin{lstlisting}[style=CStyle]
   float g = 1.32471795724474602596; //Plastische Zahl
   float a1 = (1.0/g) * frame_count;
   float a2 = (1.0/(g*g)) * frame_count;
   x[n] = (0.5 + a1*n) % 1; //toroidally shifted
   y[n] = (0.5 + a2*n) % 1; //toroidally shifted
\end{lstlisting}


%\subsection{Low Discrepency Sequenzen}
%Aus dem Vorhandensein einer low discrepency Sequenz folgt dass auch alle 
%Subsequenzen derartiger Gestalt sind: Gemessen am Anteil der Punkte innerhalb
%einer (Sub-)sequenz.

%\begin{equation}\label{eq:Low Discrepency}
%    \sup{\left| \frac{\left| s_{1}...s_{n} \cap [c,d]\right|}{N} - \frac{d-c}{b-a} \right|}
%\end{equation}

%Jedes x $\in s_{n}$ fällt mit der annähernd gleichen Wahrscheinlichkeit
%in das Subintervall [c,d]. Wäre es die selbe Wahrscheinlichkeit hätten 
%wir die Uniformität.

%\label{sec:quasi monte carlo}
%\subsection{Quasi Monte-Carlo Methode}
%Diese Methode macht sich die \nameref{eq:Low Discrepency} Folgen zu Nutze
%im Gegensatz zu der ursprünglichen Monte-Carlo Methode \ref{ch:Content1:sec:Path Tracer},
%welche auf pseudozufälligen Zahlen basiert. Im Gegensatz zu ihr haben wir 
%hier eine schnellere Konvergenz \textit{O}($\frac{1}{N}$).
%Wichtig für das weitere Verständnis ist die Randomisierung einer von Grunde
%her deterministische Low Discrepency Sequenz. Ein Verfahren, das 
%zufällige Shiften, bildet eine neue Sequenz $y_{i}$ aus $x_{i}$ durch 
%eine komponenteweise Addition mit einem zufälligen Wert w.
