Betrachten wir Verfahren wie das \nameref{sec:quasi monte carlo} Verfahren, 
so bekommen wir klassische unkorrelierte Pixelwerte. Resultat hiervon 
sind typische white noise Fehlerverteilungen. Erkenntnisse aus 
\cite{10.111712.152707} bringen die Vorteile von anderen Fehlerverteilungen
zur Geltung. Blue noise verteilte Fehlverteilungen im Bildraum schaffen hiernach 
einen besseren optischen Eindruck für das menschliche Auge. 
Die Arbeit \cite{georgiev2016blue} präsentiert zum konventionellen
zufälligen Shiften \ref{sec:quasi monte carlo} eine Alternative, die den 
gewählten Offset anhand einer über das Fenster gekachelten \nameref{ch:Content1:sec:blue noise} 
textur wählt. Im Folgenden wird auf das
BNDS (blue noise dithered sampling) und deren \textit{a priori} Eigenschaften 
eingegangen um anschließend die \nameref{ch:Content2:sec:a Posteriori} Eigenschaften des 
Temporalen Algorithmus besser zu verstehen.

Die Optimalität der 1-Dimensionalität der Sample Anzahl geht aus den 
Ausführungen in \cite[S.3]{hal02158423} hervor und hat mit der 
abnehmenden Bildraumkorrelation der Samples mit gleichzeitig steigender
Anzahl zu tun.

Die hohe Dimensionalität der verwendeten Blue Noise Textur bleibt ein offenes 
Problem \cite{bluenoisechrisschied}. Deshalb ist eine niedrige Anzahl von 
Dimensionalität günstig.
für das BNDS günstig und der temporale Algorithmus verwendet eine 
1-D Textur. 

\label{alg:Grund für homogene Flächen}

