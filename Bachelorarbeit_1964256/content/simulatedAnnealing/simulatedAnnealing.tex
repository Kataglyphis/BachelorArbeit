Im vorherigen Kapitel, dem Retargeting Schritt \ref{ch:Content2:sec:Retargeting},
wird eine vorberechnete Retargeting-Textur verwendet. Diese speichert eine
Permutation, die unsere blue noise Textur vom frame t in eine
blue noise Textur vom frame t+1 umwandelt. Diese Permutation wird 
dann auf die Startwerte angewandt bevor das nächste frame t+1 gerendert wird.
Dadurch werden die blue noise Umverteilung der Sorting Phase \ref{ch:Content2:sec:Sorting}
akkumuliert und die optische Aufwertung erst richtig sichtbar.
Die retarget Textur wird mit Hilfe von \textbf{simulated annealing} 
\cite{hal02158423} berechnet. Wir wollen somit eine approximativ 
optimale Lösung finden: Permutiere Pixel der blue noise Textur von 
frame t bis Sie sehr ähnlich verteilt sind wie die Pixel der blue noise
Textur von frame t+1. Dabei ist die Lokalität der Vertauschungen, 
welche wir bereits in der Sorting Phase\ref{ch:Content2:sec:Sorting}
verwendet haben, wichtig.

Die Funktion nach der optimiert wird ist an die Formel aus\cite{georgiev2016blue} angelehnt.
\begin{equation}\label{eq:pixelenergyfunction}
    E(M) = \sum_{p \neq q}E(p,q) = 
           \sum_{p \neq q} \mathrm{e}^{-\frac{\Vert{p_{i}-q_{i}}\Vert^{2}}{\sigma_{i}^{2}} -
           \frac{\Vert{p_{s}-q_{s}}\Vert^{d/2}}{\sigma_{s}^{2}}}
\end{equation}
Wähle nach \cite{ulichney1993void} $\sigma_{i} = 2.1$ und $\sigma_{s} = 1$ 
Zu den Pixeln p,q beschreibt $p_{i} und q_{i}$ ihre jeweiligen Koordinaten.
Und $p_{s} und q_{s}$ sind ihre d-dimensionalen Samplewerte.


\begin{algorithm}[H]
    \caption{\textbf{Simulated Annealing} finde sehr gute Lösung}
    \begin{algorithmic}[1]
        \State initialisiere Startzustand $s_{0}$
        \For{i=0...$i_{max}$}
        \State
        \EndFor
        \State return Endzustand;
    \end{algorithmic}
    \label{alg:retargeting}
\end{algorithm}
