Der nun behandelte temporale Algorithmus von \cite{hal02158423} beruht 
im Gegensatz zu \cite{georgiev2016blue} auf \textit{nachträglichen} 
Annahmen. Welches zur Folge hat, dass die Dimension unseres Path Tracers
\ref{ch:Content1:sec:PathTracer}, sowie die Stichprobenanzahl einhergehen 
mit der Verteilung der Integrationsfehler als blue noise im Bildraum.
Die zu Grunde liegenden Annahmen sollen nun im Folgenden untersucht werden.

\subsection{Theoretische Grundlage}
Im Kapitel \nameref{ch:Content1:sec:PathTracer} haben wir gesehen, dass 
wir den Wert eines Pixels (i,j) klassischerweise mit einem zufälligen
Startwert durch eine Monte-Carlo Integration erhalten. Wir betrachten im
Folgenden eine (theoretische) Menge von allen möglichen Werten eines 
Pixels, welche durch alle möglichen Startwerte generiert wurde.
In \ref{eq:pixelschaetzungPDF} ist die Wahrscheinlichkeitsdichtefunktion
$h_{ij}$ aufgetragen, als eine Funktion über alle möglichen Werte 
$I_{ij}$ eines Pixels (i,j).

\begin{equation}\label{eq:pixelschaetzungPDF}
    H_{ij}([I_{Anfang},I_{Ende}]) = \int_{I_{Anfang}}^{I_{Ende}} h_{ij} dI
\end{equation}

Daraus lässt sich die Gleichbedeutung zweier Aussagen begründen:
Das Rendern des Pixels (i,j) und das Wählen eines Pixelswertes $I_{ij}$
von unser zuvor formulierten Wahrscheinlichkeitsdichtefunktion $h_{ij}$.

\begin{equation}\label{eq:inverseFunc}
    I_{ij} = H_{ij}^{-1}(x), x \in [0,1]
\end{equation}

Nun betrachte man die Werte für x in \ref{eq:inverseFunc} als im Bildraum
blue noise verteilte Zahlen. Daraus folgt, dass die resultierenden
Integrationsfehler auch als blue noise im Bildraum verteilt sind.


\subsection{Praktische Durchführung}
Die Berechnung des vollständigen Histogramms ist für eine Echtzeitanwendung
zu kostenintensiv. Stattdessen könnte man auch die dadurch beanspruchte 
Rechenleistung auf z.B mehrere Samples pro Pixel verteilen.
Stattdessen werden wir in dem temporalen Algorithmus von \cite{hal02158423}
das Histogramm mit dem vorherigen Frame approximieren. 
Bereits vorherige Arbeiten \cite{Schied:2018:GER:3273023.3233301} haben die 
Wirksamkeit eines solchen temporalen Ansatzes(Zugriff auf das vorherige Frame)
gezeigt. Die Approximation des Histogramms erfolgt dadurch mit dem $Frame_{t}$ 
für $Frame_{t+1}$,indem umliegende Pixel in das Histogramm aufgenommen werden.
Offensichtliche Konsequenzen dieser blockweisen Verarbeitung sind schlechte blue noise 
Fehlerverteilungen im Bildraum bei sich stark ändernden Bildausschnitten
(so z.B. bei Objektkanten), da dort die Annahme, dass eine ähnliche Oberfläche
zur Farbgebung beiträgt verletzt wird.