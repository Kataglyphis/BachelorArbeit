Ideen eines temporalen Ansatzes zur Steigerung der Bildqualität im Kontext der Strahlenverfolgung werden bereits von 
aktuellen Untersuchungen verfolgt \cite{meyer2006statistical}. Ein Ansatz der Akkumulation wird in 
\cite{schied2017spatiotemporal} benutzt.

\begin{algorithm}[H]
    \caption{Beispielhafte Akkumulation}
    \begin{algorithmic}[1]
        \State Texture2D current\_frame;
        \State RWTexture2D accumulation\_buffer;
        \State float4 current\_color = current\_frame[pixel\_pos];
        \State float4 prev\_color = accumulation\_buffer[pixel\_pos];
        \State accumulation\_buffer[pixel\_pos] = (frame\_count * prev\_color + current\_color) / (frame\_count + 1);
    \end{algorithmic}
    \label{alg:TemporalAccumulation}
\end{algorithm}

Diese klassische Formulierung, verletzt unsere Annahme für Gleichung \ref{eq:inverse Funktion}
in den \nameref{ch:Content2:sec:a Posteriori}-Bedingungen des zugrundeliegenden Algorithmus 
\ref{ch:Temporaler Algorithmus}. Durch diese Akkumulierung bestimmt nicht mehr allein der 
Anfangswert die Pixelfarbe!