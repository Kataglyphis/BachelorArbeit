In diesem Abschnitt wird auf den in \cite{hal02158423} vorgestellten, temporalen Algorithmus eingegangen.
Dieser besteht grundsätzlich aus dem \nameref{ch:Content2:sec:Sorting} sowie den 
\nameref{ch:Content2:sec:Retargeting}. Es sollte unbedingt beachtet werden, dass folgende
Annahmen getroffen wurden: Der Algorithmus arbeitet Blockweise auf den Pixeln und erwartet, dass benachbarte
Pixel innerhalb dieses Blockes den selben Wert haben. Da wir einen temporalen Algorithmus haben, soll diese Annahme 
auch über mehrere gerenderte Bilder hinweg gelten. Es sollte also beachtet werden, dass der Algorithmus z.B. nicht 
für Objektkanten oder ruckartige Bewegungen (der Kamera oder Objekte) ausgelegt ist.
Des Weiteren gehen wir aus den \textit{nachträglichen Eigenschaften}
\ref{ch:Content2:sec:APosteriori} gewonnen Einsicht, dass die Wahl 
unserer Anfangswerte des \nameref{ch:Content1:sec:PathTracer} unsere 
Fehlerverteilung im Bildraum beeinflusst, aus. Somit werden wir ein Umsortieren
unserer Anfangswerte anhand einer blue noise Textur vornehmen, um so auch 
für die gerenderten Farbwerte der Pixel eine blue noise Fehlerverteilung zu 
erhalten. 


\par
\cite{bluenoisechrisschied} empfiehlt die Benutzung von $64^{2}$ 8-bit 
Texturen. Eine Benutzung in der Hinsicht, alle 64 bereitgestellten Varianten
in ein Array zu laden, jedes Frame ein neues Zufälliges zu verwenden und
mit einem zufälligen Offset drauf zuzugreifen. Die Database von Texturen 
\cite{bluenoisechrisschied} enthält für die empfohlene Auflösung jeweils
Varianten mit einer unterschiedlicher Anzahl von Kanälen. Wir wählen
die Anzahl der Kanäle anhand der Anzahl der Dimensionen, die gleichzeitig
blue noise verteilt werden sollen.



