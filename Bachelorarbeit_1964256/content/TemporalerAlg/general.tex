In diesem Abschnitt wird auf den in \cite{hal02158423} vorgestellten, temporalen Algorithmus eingegangen.
Dieser besteht grundsätzlich aus dem \nameref{ch:Content2:sec:Sorting} sowie den 
\nameref{ch:Content2:sec:Retargeting}. Es sollte unbedingt beachtet werden, dass folgende
Annahmen getroffen wurden: Der Algorithmus arbeitet Blockweise auf den Pixeln und erwartet, dass benachbarte
Pixel innerhalb dieses Blockes den selben Wert haben. Da wir einen temporalen Algorithmus haben, soll diese Annahme 
auch über mehrere gerenderte Bilder hinweg gelten. Es sollte also beachtet werden, dass der Algorithmus z.B. nicht 
für Objektkanten oder ruckartige Bewegungen (der Kamera oder Objekte) ausgelegt ist.
\cite{heitz:hal-02150657}

\cite{quasirandomsequencesbyRoberts}
\todo{find the right place for this chapter}\label{code:golden_ratio}
\begin{lstlisting}[style=CStyle]
   float g = 1.32471795724474602596; //Plastische Zahl
   float a1 = 1.0/g;
   float a2 = 1.0/(g*g);
   x[n] = (0.5+a1*n) %1; //toroidally shifted
   y[n] = (0.5+a2*n) %1; //toroidally shifted
\end{lstlisting}
Die sogenannte Plastische Zahl in \ref{code:golden_ratio} ist die Lösung der
Gleichung \ref{eq:plasticnumber}
\begin{equation}\label{eq:plasticnumber}
   x^{3} - x - 1 = 0
\end{equation}
\todo{add further information and references!!!}
