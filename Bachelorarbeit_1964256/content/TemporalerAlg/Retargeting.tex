\cite{hal02158423}

Zu Grunde liegender Sinn dieses Schrittes: Vertauschen der Anfangswerte, die 
verteilt sind wie $BlueNoise_{t}$, sodass Sie verteilt sind wie die 
$BlueNoise_{t+1}$. Aufgrund dessen haben wir eine Aufsummierung der
blue noise Fehlerverteilungen über viele Frames.

\begin{algorithm}[H]
    \caption{\textbf{Retargeting Schritt} t Vor Rendern Frame t+1 nach Sortier Schritt}
    \begin{algorithmic}[1]
        \State //permutation indices from precomputed texture
        \State $retaget_{t}$ = retargettexure[calcCorrectOffset(incomingbluenoisetexture)];
        \State List<PixelPermutation> L = $retaget_{t}$
        \For{i = 1 .. numberOfPixelsPerBlock}
        \State $retargetedSeeds(L.getNewIndices()) = incomingSeeds(L.getOldIndices());$
        \EndFor
    \end{algorithmic}
    \label{alg:retargeting}
\end{algorithm}

\begin{equation}\label{eq:pixelenergyfunction}
    E(M) = \sum_{p \neq q}E(p,q) = 
           \sum_{p \neq q} \mathrm{e}^{-\frac{\Vert{p_{i}-q_{i}}\Vert^{2}}{\sigma_{i}^{2}} -
           \frac{\Vert{p_{s}-q_{s}}\Vert^{d/2}}{\sigma_{s}^{2}}}
\end{equation}
