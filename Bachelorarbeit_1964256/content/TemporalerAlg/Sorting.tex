In diesem Abschnitt machen wir uns die getroffenen \nameref{ch:Content2:sec:a Posteriori} Annahmen 
zu Nutze. Nach dem Rendern von $Bild_{t}$ (und vor dem Rendern von $Bild_{t+1}$) nehmen wir die Sortierung
anhand der Pixelwerte von $Bild_{t}$ vor. 

\par

Die Pixel innerhalb eines Blocks (z.B. Blockgröße = 64, Auswirkung von verschiedenen Blockgrößen in Abschnitt \ref{subsec:Blockgröße}) werden anhand ihrer 
Intensität in einer aufsteigenden Liste sortiert und damit zu einem Histogramm. Dieses generierte Histogramm wird für jeden Pixel innerhalb des Blocks verwendet.
Die sortierte Liste als Histogramm zu benutzen ist zulässig. Denn beim Benutzen der inversen Funktion \ref{eq:inverse Funktion} machen wir nichts Anderes, als 
zufällige Zahlen auf ein bestimmtes Wahrscheinlichkeitsquantil, das einer gewissen Sortierung entspricht, abzubilden. Deshalb können wir die Pixelintensitäten 
innerhalb eines Blocks anhand ihrer Indizes (entspricht den Wahrscheinlichkeitsquantilen bei der Wahrscheinlichkeitsfunktion) sortieren! 

\begin{algorithm}[H]
    \caption{\textbf{Sortier Schritt t}}
    \begin{algorithmic}[1]
        \State pixel \textbf{consists of} value,index;
        \State List framePixelsIntensities, noiseIntensities;
        \State $assert(sizeof(framePixelsIntensities)==BLOCKSIZE)$;
        \State $assert(sizeof(noiseIntensities)==BLOCKSIZE)$;
        \State List L $\leftarrow$ pixels of frame t in block;
        \State \hfill
        \State //sort the two lists by means of intensities
        \State sort\_by\_means\_of\_value(framePixelsIntensities);
        \State sort\_by\_means\_of\_value(noiseIntensities);
        \State \hfill
        \State //correspondig indices are also sorted by means of values
        \For{$i = 1 .. BLOCKSIZE$}
        \State $sortedSeeds(noiseIntensities.getIndex(i)) = $
        \State $incomingSeeds(framePixelIntensities.getIndex(i))$;
        \EndFor
        \State //now the seeds are sorted following a blue noise distribution
    \end{algorithmic}
    \label{alg:Sortier}
\end{algorithm}

Das Verwenden der anhand Pixel- und blue noise Werten sortierten Indizes in Zeile 13,14 des Algorithmus \ref{alg:Sortier} ist die praktische Anwendung der zuvor eingeführten 
Quantilfunktion \ref{eq:inverse Funktion}. Denn die Anfangswerte $x \in [0,1]$ der Gleichung \ref{eq:inverse Funktion} werden hier einem eindeutigen sortierten Intensitätswert 
zugeteilt. Das Abbilden der Pixelwerte auf Werte einer \nameref{ch:Content1:sec:blue noise} Textur garantiert nun die entsprechende Korrelation 
der Anfangswerte zueinander (siehe auch Abschnitt \ref{ch:Content1:sec:blue noise sampling}).
Die Annahme der Lokalität, also der Homogenität (in der Abbildung \ref{fig:Pixelblöcke} demonstriert) einer Fläche innerhalb eines Blocks, 
erlaubt die Approximation des Histogramms \ref{fig:Pixelwerte} eines Pixels anhand der Intensitäten aller anderen Pixel innerhalb des Blocks.
Wird der Block zu groß gewählt, so wir die Approximation zu rechenintensiv die parallele Ausführbarkeit 
und die  Homogenitätseigenschaft (Bildkohärenz) gehen verloren, wohingegen zu kleine Blockgrößen nur eine sehr wage und ungenaue Approximation des 
Histogramms \ref{pic:histogramOfEstimates} liefern.

\par

Hierbei muss noch eine wichtige Anmerkung gemacht werden. Die Fehlerverteilung der Pixelwerte im Bildraum konvergiert auf diese Weise nicht zu einer 
blue noise Verteilung, denn wir wechseln in jedem Bild die verwendeten blue noise Texturen(theoretisch, praktischerweise werden wir hier eine
Textur verwenden und mit Erkenntnissen aus \nameref{ch:Content1:sec:Quasi-Zufallsfolgen}) quasi-zufällig zugreifen um so einen solchen Effekt zu erreichen). 
Dieser Schritt alleine reicht also nicht für den erwünschten Effekt zu erreichen.

\label{subsec:Blockgröße}
\subsubsection{Blockgöße}

Für den Sortierschritt gibt es ein Abwägung zu treffen: Hohe Blockgröße B bedeutet eine bessere Approximation 
des Histogramms (siehe Bild \ref{pic:histogramOfEstimates}), wohingegen allerdings die Raum-Zeit 
Kohärenz verloren geht.

\begin{figure}[H]

    %%%%%%%%%%%%%%%%%%%%%%%%%%%%%%%%%%%%%%%%%%%%%%%%%%%%%%%%%%%%%%%%%%%%%%%%%%%%%%%%%%%%%%%%%%%%%%%%%%%%%%
    %%%%%%%%%%%%%%%%%%%%%%%%%%%%%%% first row 
    %%%%%%%%%%%%%%%%%%%%%%%%%%%%%%%%%%%%%%%%%%%%%%%%%%%%%%%%%%%%%%%%%%%%%%%%%%%%%%%%%%%%%%%%%%%%%%%%%%%%%%

    \centering
    \begin{subfigure}[b]{0.2\linewidth}
      \includegraphics[width=\linewidth]{content/TemporalerAlg/Bilder/Sorting/DiffDimensions/2/seed_debug_5.0_small.png}
       \caption{FT B = 2}
       \label{pic:fftB_2}
    \end{subfigure}
    \begin{subfigure}[b]{0.2\linewidth}
      \includegraphics[width=\linewidth]{content/TemporalerAlg/Bilder/Sorting/DiffDimensions/3/seed_debug_5.0_small.png}
      \caption{FT B = 3}
      \label{pic:fftB_3}
    \end{subfigure}
    \begin{subfigure}[b]{0.2\linewidth}
      \includegraphics[width=\linewidth]{content/TemporalerAlg/Bilder/Sorting/DiffDimensions/4/seed_debug_5.0_small.png}
      \caption{FT B = 4}
      \label{pic:fftB_4}
    \end{subfigure}
    \begin{subfigure}[b]{0.2\linewidth}
        \includegraphics[width=\linewidth]{content/TemporalerAlg/Bilder/Sorting/DiffDimensions/5/seed_debug_5.0_small.png}
        \caption{FT B = 5}
        \label{pic:fftB_5}
    \end{subfigure}
    
    %%%%%%%%%%%%%%%%%%%%%%%%%%%%%%%%%%%%%%%%%%%%%%%%%%%%%%%%%%%%%%%%%%%%%%%%%%%%%%%%%%%%%%%%%%%%%%%%%%%%%%
    %%%%%%%%%%%%%%%%%%%%%%%%%%%%%%% second row
    %%%%%%%%%%%%%%%%%%%%%%%%%%%%%%%%%%%%%%%%%%%%%%%%%%%%%%%%%%%%%%%%%%%%%%%%%%%%%%%%%%%%%%%%%%%%%%%%%%%%%%

    \begin{subfigure}[b]{0.2\linewidth}
        \includegraphics[width=\linewidth]{content/TemporalerAlg/Bilder/Sorting/DiffDimensions/2/seed_debug_5.0_small_screen.png}
         \caption{Ausschnitt B=2}
         \label{pic:screen_B2}
    \end{subfigure}
    \begin{subfigure}[b]{0.2\linewidth}
        \includegraphics[width=\linewidth]{content/TemporalerAlg/Bilder/Sorting/DiffDimensions/3/seed_debug_5.0_small_screen.png}
         \caption{Ausschnitt B=3}
         \label{pic:screen_B3}
    \end{subfigure}
    \begin{subfigure}[b]{0.2\linewidth}
        \includegraphics[width=\linewidth]{content/TemporalerAlg/Bilder/Sorting/DiffDimensions/4/seed_debug_5.0_small_screen.png}
         \caption{Ausschnitt B=4}
         \label{pic:screen_B4}
    \end{subfigure}
    \begin{subfigure}[b]{0.2\linewidth}
        \includegraphics[width=\linewidth]{content/TemporalerAlg/Bilder/Sorting/DiffDimensions/5/seed_debug_5.0_small_screen.png}
         \caption{Ausschnitt B=5}
         \label{pic:screen_B5}
    \end{subfigure}

    %%%%%%%%%%%%%%%%%%%%%%%%%%%%%%%%%%%%%%%%%%%%%%%%%%%%%%%%%%%%%%%%%%%%%%%%%%%%%%%%%%%%%%%%%%%%%%%%%%%%%%
    %%%%%%%%%%%%%%%%%%%%%%%%%%%%%%% third row 
    %%%%%%%%%%%%%%%%%%%%%%%%%%%%%%%%%%%%%%%%%%%%%%%%%%%%%%%%%%%%%%%%%%%%%%%%%%%%%%%%%%%%%%%%%%%%%%%%%%%%%%

    \centering
    \begin{subfigure}[b]{0.2\linewidth}
      \includegraphics[width=\linewidth]{content/TemporalerAlg/Bilder/Sorting/DiffDimensions/6/seed_debug_5.0_small.png}
       \caption{FT B = 6}
       \label{pic:fftB_6}
    \end{subfigure}
    \begin{subfigure}[b]{0.2\linewidth}
      \includegraphics[width=\linewidth]{content/TemporalerAlg/Bilder/Sorting/DiffDimensions/7/seed_debug_5.0_small.png}
      \caption{FT B = 7}
      \label{pic:fftB_7}
    \end{subfigure}
    \begin{subfigure}[b]{0.2\linewidth}
      \includegraphics[width=\linewidth]{content/TemporalerAlg/Bilder/Sorting/DiffDimensions/8/seed_debug_5.0_small.png}
      \caption{FT B = 8}
      \label{pic:fftB_8}
    \end{subfigure}
    \begin{subfigure}[b]{0.2\linewidth}
        \includegraphics[width=\linewidth]{content/TemporalerAlg/Bilder/Sorting/DiffDimensions/10/seed_debug_5.0_small.png}
        \caption{FT B = 10}
        \label{pic:fftB_10}
    \end{subfigure}
    
    %%%%%%%%%%%%%%%%%%%%%%%%%%%%%%%%%%%%%%%%%%%%%%%%%%%%%%%%%%%%%%%%%%%%%%%%%%%%%%%%%%%%%%%%%%%%%%%%%%%%%%
    %%%%%%%%%%%%%%%%%%%%%%%%%%%%%%% 4th row
    %%%%%%%%%%%%%%%%%%%%%%%%%%%%%%%%%%%%%%%%%%%%%%%%%%%%%%%%%%%%%%%%%%%%%%%%%%%%%%%%%%%%%%%%%%%%%%%%%%%%%%

    \begin{subfigure}[b]{0.2\linewidth}
        \includegraphics[width=\linewidth]{content/TemporalerAlg/Bilder/Sorting/DiffDimensions/6/seed_debug_5.0_small_screen.png}
         \caption{Ausschnitt B=6}
         \label{pic:screen_B6}
    \end{subfigure}
    \begin{subfigure}[b]{0.2\linewidth}
        \includegraphics[width=\linewidth]{content/TemporalerAlg/Bilder/Sorting/DiffDimensions/7/seed_debug_5.0_small_screen.png}
         \caption{Ausschnitt B=7}
         \label{pic:screen_B7}
    \end{subfigure}
    \begin{subfigure}[b]{0.2\linewidth}
        \includegraphics[width=\linewidth]{content/TemporalerAlg/Bilder/Sorting/DiffDimensions/8/seed_debug_5.0_small_screen.png}
         \caption{Ausschnitt B=8}
         \label{pic:screen_B8}
    \end{subfigure}
    \begin{subfigure}[b]{0.2\linewidth}
        \includegraphics[width=\linewidth]{content/TemporalerAlg/Bilder/Sorting/DiffDimensions/10/seed_debug_5.0_small_screen.png}
         \caption{Ausschnitt B=10}
         \label{pic:screen_B10}
    \end{subfigure}

    \caption{Verschiedene Blockgrößen B bei selben homogenen Szenenausschnitt; jeweils drittes Bild}
    \label{fig:VerschiedeneBlockgrößenSorting}
      
\end{figure}

Generell lässt sich aus den Untersuchungen in Abbildung \ref{fig:VerschiedeneBlockgrößenSorting} schließen,
dass eine geringe Blockgröße des Sortierschrittes ein Fehlen bzw. starke Abschwächung der 
\nameref{ch:Content1:sec:blue noise} Fehlerverteilung im Bildraum zur Folge hat. Erklärung: Das Histogramm
(siehe \ref{pic:histogramOfEstimates}), Wahrscheinlichkeitsfunktion
aller möglichen Pixelwerte, wird mit zu wenigen Werten approximiert. Diese sehr wage Approximation 
hat zur Folge, dass die Sortierung einer randomisierten Folge entspricht. Deshalb sind im Bildraum
viele Cluster und die white noise Eigenschaften in den Spektra zu erkennen.

%%%%%%%%%%%%%%%%%%%%%%%%%%%%%%%%%%%%%%%%%%%%%%%%%%%%%%%%%%%%%%%%%%%%%%%%%%%%%%%%%%%%%%%%%%%%%%%%%%%%%%
%%%%%%%%%%%%%%%%%%%%%%%%%%%%%%% Screenshotreihe
%%%%%%%%%%%%%%%%%%%%%%%%%%%%%%%%%%%%%%%%%%%%%%%%%%%%%%%%%%%%%%%%%%%%%%%%%%%%%%%%%%%%%%%%%%%%%%%%%%%%%%

\newpage

\begin{figure}[H]

    \begin{subfigure}{\textwidth}
        \centering \includegraphics[scale=.25]{content/TemporalerAlg/Bilder/Sorting/Szene/Szene1.png}
        \caption{Szene}
        \label{fig:Nur_Sorting_Szene_t1}
    \end{subfigure}
    \begin{subfigure}{0.5\textwidth}
        \centering \includegraphics[width=0.4\linewidth]{content/TemporalerAlg/Bilder/Sorting/Ausschnitte/Ausschnitt1.png} 
        \caption{Szenenausschnitt}
        \label{fig:Nur_Sorting_ausschnitt_t1}
    \end{subfigure}
    \begin{subfigure}{0.5\textwidth}
        \centering \includegraphics[width=0.4\linewidth]{content/TemporalerAlg/Bilder/Sorting/Spektren/Ausschnitt1.png}
        \caption{Fouriertransformierte des Ausschnitts}
        \label{fig:Nur_Sorting_Fouriertransformierte_t1}
    \end{subfigure}
        \caption{Zeitpunkt t=1}
        \label{fig:Nur_Sorting_Verlauf_t1}
\end{figure}

\begin{figure}[H]
    \begin{subfigure}{\textwidth}
        \centering \includegraphics[scale=.25]{content/TemporalerAlg/Bilder/Sorting/Szene/Szene2.png}
        \caption{Szene}
        \label{fig:Nur_Sorting_Szene_t2}
    \end{subfigure}
    \begin{subfigure}{0.5\textwidth}
        \centering\includegraphics[width=0.4\linewidth]{content/TemporalerAlg/Bilder/Sorting/Ausschnitte/Ausschnitt2.png} 
        \caption{Szenenausschnitt}
        \label{fig:Nur_Sorting_ausschnitt_t2}
    \end{subfigure}
    \begin{subfigure}{0.5\textwidth}
        \centering\includegraphics[width=0.4\linewidth]{content/TemporalerAlg/Bilder/Sorting//Spektren/Ausschnitt2.png}
        \caption{Fouriertransformierte des Ausschnitts}
        \label{fig:Nur_Sorting_Fouriertransformierte_t2}
    \end{subfigure}
        \caption{Zeitpunkt t=2}
        \label{fig:Nur_Sorting_Verlauf_t2}
\end{figure}

\begin{figure}[H]
    \begin{subfigure}{\textwidth}
        \centering \includegraphics[scale=.25]{content/TemporalerAlg/Bilder/Sorting/Szene/Szene3.png}
        \caption{Szene}
        \label{fig:Nur_Sorting_Szene_t3}
    \end{subfigure}
    \begin{subfigure}{0.5\textwidth}
        \centering\includegraphics[width=0.4\linewidth]{content/TemporalerAlg/Bilder/Sorting/Ausschnitte/Ausschnitt3.png} 
        \caption{Szenenausschnitt}
        \label{fig:Nur_Sorting_ausschnitt_t3}
    \end{subfigure}
    \begin{subfigure}{0.5\textwidth}
        \centering\includegraphics[width=0.4\linewidth]{content/TemporalerAlg/Bilder/Sorting/Spektren/Ausschnitt3.png}
        \caption{Fouriertransformierte des Ausschnitts}
        \label{fig:Nur_Sorting_Fouriertransformierte_t3}
    \end{subfigure}
        \caption{Zeitpunkt t=3}
        \label{fig:Nur_Sorting_Verlauf_t3}
\end{figure}

\begin{figure}[H]
    \begin{subfigure}{\textwidth}  
        \centering \includegraphics[scale=.25]{content/TemporalerAlg/Bilder/Sorting/Szene/Szene4.png}
        \caption{Szene}
        \label{fig:Nur_Sorting_Szene_t4}
    \end{subfigure}
    \begin{subfigure}{0.5\textwidth}
        \centering\includegraphics[width=0.4\linewidth]{content/TemporalerAlg/Bilder/Sorting/Ausschnitte/Ausschnitt4.png} 
        \caption{Szenenausschnitt}
        \label{fig:Nur_Sorting_ausschnitt_t4}
    \end{subfigure}
    \begin{subfigure}{0.5\textwidth}
        \centering\includegraphics[width=0.4\linewidth]{content/TemporalerAlg/Bilder/Sorting//Spektren/Ausschnitt4.png}
        \caption{Fouriertransformierte des Ausschnitts}
        \label{fig:Nur_Sorting_Fouriertransformierte_t4}
    \end{subfigure}
        \caption{Zeitpunkt t=4}
        \label{fig:Nur_Sorting_Verlauf_t4}
\end{figure}

\begin{figure}[H]
    \begin{subfigure}{\textwidth}   
        \centering \includegraphics[scale=.25]{content/TemporalerAlg/Bilder/Sorting/Szene/Szene5.png}
        \caption{Szene}
        \label{fig:Nur_Sorting_Szene_t5}
    \end{subfigure}
    \begin{subfigure}{0.5\textwidth}
        \centering\includegraphics[width=0.4\linewidth]{content/TemporalerAlg/Bilder/Sorting/Ausschnitte/Ausschnitt5.png} 
        \caption{Szenenausschnitt}
        \label{fig:Nur_Sorting_ausschnitt_t5}
    \end{subfigure}
    \begin{subfigure}{0.5\textwidth}
        \centering\includegraphics[width=0.4\linewidth]{content/TemporalerAlg/Bilder/Sorting//Spektren/Ausschnitt5.png}
        \caption{Fouriertransformierte des Ausschnitts}
        \label{fig:Nur_Sorting_Fouriertransformierte_t5}
    \end{subfigure}
        \caption{Zeitpunkt t=5}
        \label{fig:Nur_Sorting_Verlauf_t5}
\end{figure}

\begin{figure}[H]
    \begin{subfigure}{\textwidth}   
        \centering \includegraphics[scale=.25]{content/TemporalerAlg/Bilder/Sorting/Szene/Szene6.png}
        \caption{Szene}
        \label{fig:Nur_Sorting_Szene_t6}
    \end{subfigure}
    \begin{subfigure}{0.5\textwidth}
        \centering\includegraphics[width=0.4\linewidth]{content/TemporalerAlg/Bilder/Sorting/Ausschnitte/Ausschnitt6.png} 
        \caption{Szenenausschnitt}
        \label{fig:Nur_Sorting_ausschnitt_t6}
    \end{subfigure}
    \begin{subfigure}{0.5\textwidth}
        \centering\includegraphics[width=0.4\linewidth]{content/TemporalerAlg/Bilder/Sorting/Spektren/Ausschnitt6.png}
        \caption{Fouriertransformierte des Ausschnitts}
        \label{fig:Nur_Sorting_Fouriertransformierte_t6}
    \end{subfigure}
        \caption{Zeitpunkt t=6}
        \label{fig:Nur_Sorting_Verlauf_t6}
\end{figure}

\begin{figure}[H]
    \begin{subfigure}{\textwidth}   
        \centering \includegraphics[scale=.25]{content/TemporalerAlg/Bilder/Sorting/Szene/Szene7.png}
        \caption{Szene}
        \label{fig:Nur_Sorting_Szene_t7}
    \end{subfigure}
    \begin{subfigure}{0.5\textwidth}
        \centering\includegraphics[width=0.4\linewidth]{content/TemporalerAlg/Bilder/Sorting/Ausschnitte/Ausschnitt7.png} 
        \caption{Szenenausschnitt}
        \label{fig:Nur_Sorting_ausschnitt_t7}
    \end{subfigure}
    \begin{subfigure}{0.5\textwidth}
        \centering\includegraphics[width=0.4\linewidth]{content/TemporalerAlg/Bilder/Sorting/Spektren/Ausschnitt7.png}
        \caption{Fouriertransformierte des Ausschnitts}
        \label{fig:Nur_Sorting_Fouriertransformierte_t7}
    \end{subfigure}
        \caption{Zeitpunkt t=7}
        \label{fig:Nur_Sorting_Verlauf_t7}
\end{figure}

Die Bildreihe zeigt die ersten sieben erstellten Bilder mit ausschließlichem 
Sortieren ohne \nameref{ch:Content2:sec:Retargeting}. Bereits im dritten Bild 
\ref{fig:Nur_Sorting_Verlauf_t3} ist eine \nameref{ch:Content1:sec:blue noise}
Fehlerverteilung im Bildraum zu erkennen. Die darauffolgenden Ausschnitte 
\ref{fig:Nur_Sorting_Verlauf_t4} - \ref{fig:Nur_Sorting_Verlauf_t7}