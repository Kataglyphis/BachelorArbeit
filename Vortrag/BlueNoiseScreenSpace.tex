\documentclass[t]{beamer}
%\documentclass{beamer}
\listfiles

\mode<presentation>
{
%  \usetheme{Frankfurt}
  \usetheme[deutsch,titlepage0]{KIT}
% \usetheme[usefoot]{KIT}
% \usetheme{KIT}

  \usefonttheme{structurebold}

  \setbeamercovered{transparent}

  %\setbeamertemplate{enumerate items}[circle]
  \setbeamertemplate{enumerate items}[ball]

}
\usepackage{babel}
\date{\today}
%\DateText

\newlength{\Ku}
\setlength{\Ku}{1.43375pt}

\usepackage[utf8]{inputenc}
\usepackage[TS1,T1]{fontenc}
\usepackage{array}
\usepackage{multicol}
\usepackage{lipsum}
\usepackage{hyperref}

\usenavigationsymbols
%\usenavigationsymbols[sfHhdb]
%\usenavigationsymbols[sfhHb]

\title[]{Zeitlich stabile blue noise Fehlerverteilung im Bildraum für Echtzeitanwendungen}
\subtitle{Real-Time Raytracing}

\author[Jonas Heinle]{Jonas Heinle}

\AuthorTitleSep{\relax}

\institute[IVD - Institut für Visualisierung und Datenanalyse]{KARLSRUHER INSTITUT FÜR TECHNOLOGIE (KIT)}

\TitleImage[width=\titleimagewd]{Bilder/HDR_L_0.png}

\newlength{\tmplen}

\newcommand{\verysmall}{\fontsize{6pt}{8.6pt}\selectfont}

\begin{document}

\begin{frame}
  \maketitle
\end{frame}

\begin{frame}
  \frametitle{Blue Noise}

  
\end{frame}

\begin{frame}
  \frametitle{Simulated Annealing}

  
\end{frame}

\begin{frame}
    \frametitle{Temporaler Algorithmus}
  
    
\end{frame}

\begin{frame}
    \frametitle{A Posteriori}
  
    
\end{frame}

\begin{frame}
    \frametitle{Sorting}
  
    
\end{frame}

\begin{frame}
    \frametitle{Retargeting}
  
    
\end{frame}

\begin{frame}
    \frametitle{Temporales Projizieren}
  
    
\end{frame}


\end{document}
