Um ungewollten strukturellen Artefakten entgegenzuwirken, wird empfohlen (so auch in \cite{bluenoisechrisschied})
mehrere verschiedene \nameref{ch:Content1:sec:blue noise} Texturen in ein Array zu laden und diese in aufeinanderfolgenden
Zeitpunkten randomisiert zu verwenden. Wir führen hier Kenntnisse über Quasi-Zufallsfolgen ein (siehe auch \cite{quasirandomsequencesbyRoberts}) 
um aus den zuvor (nach Abbildung \ref{pic:tiled blue noise}) genannten Gründen nur eine kleine \nameref{ch:Content1:sec:blue noise} Textur verwenden zu können.\par
\newpage
Quasi-zufällige Sequenzen mit niedriger Abweichung sind deterministisch erzeugte Sequenzen, welche die Likelihood-Funktion der Clusterbildung

\begin{tcolorbox}[rightrule=3mm, rounded corners=east]
    \begin{equation}\label{eq:Likeli-Hood-Gleichung}
        L_{x}(\delta) = f_{\delta}(x)
    \end{equation}
\end{tcolorbox}

minimieren (siehe Abschnitt \ref{ch:Content1:sec:blue noise:Niedrige Frequenzen}) und dabei die 
Uniformität (siehe Abschnitt \ref{ch:Content1:sec:blue noise:Uniformität}) erhalten. Beide Eigenschaften 
haben wir bereits in dem \nameref{ch:Content1:sec:blue noise} Abschnitt behandelt.
Auf quasi-zufällige Zahlenfolgen angewandt bedeutet das: Wir benutzen den gesamten Raum an Zufallszahlen
\textit{uniform} und hohe Frequenzen vermeiden Regionen, aus den viele Punkte kommen, die wieder 
punktarme Regionen zur Folge hätten.   
Im Folgenden wird für uns das Ziel sein, den quasi-zufälligen Zugriff auf eine Textur 
mit zwei Dimensionen (siehe Abschnitt \ref{subsec:2-Dimensionen}) zu verstehen und nähern 
uns dabei über den Fall der Eindimensionalität (siehe Abschnitt \ref{subsec:1-Dimension}).
Ein Weg quasi-Zufallsfolgen zu beschreiben sind die zugrundeliegenden Parameter. 
Wir werden uns hier Folgen anschauen, die als Basisparameter den 
goldenen Schnitt (Gleichung \ref{eq:GoldenerSchnitt}) verwenden.

\subsubsection{Goldener Schnitt}

Der goldene Schnitt und die Generalisierung zur plastischen Zahl (Gleichung \ref{eq:plastische Zahl})
ist mitsamt ihren Eigenschaften bereits früh beschrieben worden \cite{vanderlaanplasticnumber}.
Als wichtiges Seitenverhältnis in der Architektur konnte Sie durch verschiedene Arbeiten ihren 
Eingang in die Mathematik finden \cite{krcadinac2006new}.

\begin{tcolorbox}[rightrule=3mm, rounded corners=east]
    \begin{equation}\label{eq:GoldenerSchnitt}
        \Phi_{1}^{2} - \Phi_{1} - 1 = 0 \Longrightarrow \Phi_{1} \approx 1.6180339887
    \end{equation}
\end{tcolorbox}

\label{subsec:1-Dimension}
\subsubsection{1-Dimension}
Wir benutzen Rekurrenz Sequenzen, basierend auf irrationalem 
Bruchrechnen der Form

\begin{tcolorbox}[rightrule=3mm, rounded corners=east]
    \begin{equation}\label{eq:Rekurrenz Sequenz}
        R_{1}(\alpha) : t_n = s_0 + n\alpha(mod 1); n = 1,2,3,...
    \end{equation}
\end{tcolorbox}

wobei $\alpha \in \mathbb{I}$ und das (mod 1) einen \textit{"toroidally shift"}
bezeichnet. Konkret bedeutet das, dass man für jedes neu berechnete $t_{n}$ den Bereich vor dem Dezimalpunkt abschneidet 
und den Bereich danach weiterführt. Will man mit dieser Formel eine Sequenz mit möglichst geringer
Abweichung schaffen, so wählen wir den goldenen Schnitt \ref{eq:GoldenerSchnitt}.

\label{subsec:2-Dimensionen}
\subsubsection{2-Dimensionen}

Da uns die Generalisierung des goldenen Schnittes auf die Lösung der Gleichung
\ref{eq:GeneralisierungGoldenerSchnitt} führt

\begin{tcolorbox}[rightrule=3mm, rounded corners=east]
    \begin{equation}\label{eq:GeneralisierungGoldenerSchnitt}
        x^{d+1} = x+1
    \end{equation}
\end{tcolorbox}

ist das Lösen der kubischen Gleichung \ref{eq:kubisch}

\begin{tcolorbox}[rightrule=3mm, rounded corners=east]
    \begin{equation}\label{eq:kubisch}
        x^{3} - x - 1 = 0
    \end{equation}
\end{tcolorbox}

für den zweidimensionalen Fall nötig. Die Generalisierung und Erweiterung des goldenen 
Schnittes wurde bereits ausgiebig erforscht \cite{krcadinac2006new}.

Die sogenannte Plastische Zahl in ist die Lösung der
Gleichung \ref{eq:kubisch}

\begin{tcolorbox}[rightrule=3mm, rounded corners=east]
    \begin{equation}\label{eq:plastische Zahl}
        \Phi_{2} \approx 1.32471795724
    \end{equation}
\end{tcolorbox}

Die eindimensionale Rekurrenzsequenz \ref{eq:Rekurrenz Sequenz} ist einfach erweiterbar 
für höhere Dimensionen.
\begin{tcolorbox}[rightrule=3mm, rounded corners=east]
    \begin{equation}\label{eq:1 zu N - Dimensional}
        t_{n} = n\alpha(mod 1), n = 1,2,3,..
        \alpha = (\frac{1}{\Phi_{d}}, \frac{1}{\Phi_{d}^{2}}),
    \end{equation}
\end{tcolorbox}

Für den Texturzugriff in unserem Shader bei dem temporalen Algorithmus \ref{ch:Temporaler Algorithmus}
werden wir also wie folgt vorgehen:

\begin{lstlisting}[style=CStyle]
   float g = 1.32471795724474602596; //Plastische Zahl
   int a1 = (1.0/g) * blue_noise_mask_width * frame_count;
   int a2 = (1.0/(g*g)) * blue_noise_mask_height * frame_count;
   int2 offset = (a1,a2);
   int2 new_index = offset + old_index;
   new_index.x = new_index.x % blue_noise_mask_width; //toroidally shifted
   new_index.y = new_index.y % blue_noise_mask_height; //toroidally shifted
\end{lstlisting}
