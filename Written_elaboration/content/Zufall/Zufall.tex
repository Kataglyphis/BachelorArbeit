\subsection{Pseudozufallszahlen}
\label{ch:Content1:sec:Pseudozufallszahlen}
In dieser Arbeit machen wir Gebrauch von deterministisch erzeugten, zufällig erscheinenden Zahlensequenzen (Abschnitt \ref{ch:Implementierung} für Umsetzung).
Sie erscheinen zufällig, verletzen jedoch gewissen Aspekte echter Zufallszahlen. So werden von einem Generator solcher pseudo-zufälligen Zahlen, bei selben 
Anfangswert, gleiche Zahlensequenzen erzeugt. Für unsere Anforderungen langt der Pseudozufall mit seiner überzeugenden Schnelligkeit des Generierungsprozesses.

\subsection{Uniforme Wahrscheinlichkeitsverteilung}
An einigen Stellen machen wir uns die Uniformität von Wahrscheinlichkeitsverteilungen zu nutze.
Dabei ist vereinfacht gemeint: Eine Grundmenge G, die Elemente
im Intervall [a,b] besitzt die  auf diesem Intervall gleichverteilt sind,
wird bei einem (pseudo)zufälligen Zugriff jedes Element gleichwahrscheinlich auswerfen.

%% ===========================
\subsection{Quasi-Zufallsfolgen}
\label{ch:Content1:sec:Quasi-Zufallsfolgen}
Sobol \cite{owen1998scrambling} \cite{heitz:hal-02150657} \cite{quasirandomsequencesbyRoberts}
\todo{find appropriate information and add it}

%% ===========================