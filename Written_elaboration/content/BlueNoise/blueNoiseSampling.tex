\label{subsec:dither sampling}
Nachdem wir die Eigenschaften der \nameref{ch:Content1:sec:blue noise} kennengelernt haben,
können wir zusammen mit dem Verständnis über den \nameref{ch:Content1:sec:Path Tracer} und der 
zugrundeliegenden Monte-Carlo Integration (siehe Gleichung \ref{eq:Monte-Carlo}) das \glqq dithered sampling\grqq{} verstehen.
\textit{Dithering} ist das bewusste Einbringen eines Rauschens, um entstehende Quantisierungsfehler zu randomisieren.
Klassischerweise wird eine zweidimensionale \nameref{ch:Content1:sec:blue noise} Textur verwendet, um mit einer darauf aufbauenden Schwellenwertbildung,
dieses Rauschen in ein Bild zu bringen. Hier wollen wir durch Dithering die Verteilung des entstehenden Monte-Carlo Integrationsfehlers (Rauschen) verändern.\par
Grundlage des n-dimensionalen Path Tracers werden sowohl eine \nameref{ch:Content1:sec:blue noise}-Verteilung als auch Anfangswerte $s_{n}$ mit d-Dimensionen.
Damit konkretisiert sich die Monte-Carlo Integration (siehe Gleichung \ref{eq:Monte-Carlo}) mit Integrand f zu folgender Gleichung:

\begin{tcolorbox}[rightrule=3mm, rounded corners=east]
    \begin{equation}\label{eq:concreteMonteCarlo}
        \frac{1}{N}\sum_{n=0}^{N-1}f(s_{n})
    \end{equation}
\end{tcolorbox}


Mit dem Ziel, eine blue noise Fehlerverteilung im Bildraum zu erreichen, hat sich bereits die Arbeit \cite{georgiev2016blue} beschäftigt. 
Mit im Vorraus (\glqq a priori\grqq{}) blue noise korrelierten Zahlenfolgen $s_{n}$ hoffte man ebenso korrelierte Pixelwerte nach 
der Integration zu erhalten. Der temporale Algorithmus im Kapitel \ref{ch:Temporaler Algorithmus} führt \nameref{ch:Content2:sec:a Posteriori}
Formulierungen ein (\glqq a Posteriori\grqq{}(im Nachhinein) im Sinne, dass die Methode auf bereits erzeugten Bilddaten arbeitet) und mit ihr eine 
inverse Funktion \ref{eq:inverse Funktion}, welche (approximative) garantierte korrelationerhaltene Integranden liefert!  
Neben der Eigenschaft der Verteilungsbeibehaltung der Integranden zeigt die frühere Arbeit von \cite[Seite 3]{hal02158423} eine weitere wichtige Vorraussetzung 
des blue noise sampling: Die Bildraumkohärenz. Wie auch beim konventionellen Prozess des Dithering so sollten für ein gutes Ergebnis zwei benachbarte 
Pixel einen ähnlichen Wert aufweisen \cite{3288}.




