Zur Umsetzung der bereits eingeführten Ansätze und Algorithmen benutzen wir folgende Frameworks/Bibliotheken. 

\subsection{Falcor}

Zur Umsetzung des \nameref{ch:Content1:sec:Path Tracer} benutzen wir das Framework Falcor \cite{Benty18}. Dieses Framework bietet bereits eine minimalistische Implementierung 
eines \nameref{ch:Content1:sec:Path Tracer} mit den neuen DirectX12 \ref{pic:DirectXRaytracingPipeline} Shadereinheiten, die wir für unsere Zwecke angepasst haben. 
Die hier benutzte Version 3.2.1 generiert threads, die eine falsche ID besitzen (y-Komponente < 0). Dies falschen ID's führen bei einem Texturzugriff auf nicht erlaubten
Speicherzugriff und damit zu schwarzen Bildbereichen. Diese schwarzen/falschen Bildpixel zerstören die geforderte Bedingung einer Permutation in unserem temporalen Algorithmus! 
Werden diese threads mit falscher ID nicht abgefangen, breitet sich der Fehler auf das gesamte Bild aus.\par
Mit Falcor benutzen wir das Texturenformat BGRA8Unorm. Laden wir also eine Textur haben wir es mit 8-bit Informationen pro Kanal, die im Intervall [0,1] liegen, zu tun (lineare
Abbildung $\rightarrow$ 255 wird auf 1 abgebildet).

\subsection{Simulated Annealing}

\subsection{Anfangswerte}
Für unsere vorberechnete Anfangswerttextur benötigen wir pro Eintrag zufällig generierte 32-bit Zahlen. Die Umsetzung mit Mersenne-Twister \cite{MersenneTwister} sowie mit 
der WangHash-Methode \cite{wanghash} führten zu guten Ergebnissen.

\subsection{FreeImage}
Ist als Bibliothek \cite{FreeImage} zum Laden und Speichern von Texturen geeignet. Man sollte unbedingt darauf achten, dass Bilder von links unten beginnend 
indiziert werden.

\subsection{Programmiersprache}
Das Real-Time Rendering Framework Falcor arbeitet mit C/C++ und HLSL für die Shader. Das Simulated Annealing haben wir auch mit VS2019, C++ umgesetzt.

\subsection{Visualisierungen}
Die 2D-Schaubilder wurden mit einem Wrapper für Matplotlib \cite{matplotlibwrapper} erstellt. Für 3D-Schaubilder jedoch aufgrund von aktuellen Problemen
der Software nicht. Dafür haben wir die anfallenden Datentripel in einer Textdatei abgespeichert und mit \cite{gnuplot} visualisiert. 