Da unsere Ressourcen beschränkt sind und trotz Hardwarebeschleunigung immer noch viel Rechenzeit eines Frames auf die globale Beleuchtung entfällt,
ist es von großer Bedeutung, dass unser temporaler Algorithmus keinen signifikanten zusätzlichen Aufwand schafft.
Mit einem Großteil der Rechenzeit, der auf die Berechnung des GBuffers und der globalen Beleuchtung fällt sind wir hingegen 
mit den Schritten Sorting und Retargeting sowohl auf  CPU als auf GPU Seite im niedrigen einstelligen Prozentbereich.

\begin{center}
    \begin{table}[H]
        \begin{tabular}{ | p{4cm} | p{5cm} |l |}
        \hline
        \textbf{Pipelinestage}                              &  \textbf{Rechenzeit(ms/\%)} CPU & \textbf{Rechenzeit(ms/\%)} GPU \\ \hline
        Gesamt                                              &  29.87/100\%                    & 17.76/100\%\\ \hline
        GBuffer                                             &  06.48/21.7\%                   & 01.30/7.32\%\\ \hline
        Retargeting                                         &  01.12/3.7\%                    & 00.33/1.9\%\\ \hline
        Retargeting + zusätzliches temporales Projezieren   &  01.12/3.7\%                    & 00.33/1.9\%\\ \hline
        GGXGlobalIllumination                               &  21.20/70.97\%                  & 15.51/87,33\%\\ \hline
        Sorting                                             &  00.94/3,14\%                   & 00.63/3,55\%\\ \hline
        \hline
        \end{tabular}
        \caption{Rechenzeiten die auf die einzelnen Stages fallen}
        \medskip
        \small
        * Hardware: AMD Ryzen 5 2600X, NVIDIA GeForce RTX 2060 SUPER
    \end{table}
\end{center}