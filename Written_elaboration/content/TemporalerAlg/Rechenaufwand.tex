Da unsere Ressourcen beschränkt sind und trotz Hardwarebeschleunigung immer noch viel Rechenzeit eines Frames auf die globale Beleuchtung entfällt,
ist es von Bedeutung, dass unser temporaler Algorithmus keinen signifikanten zusätzlichen Aufwand schafft.
Mit einem Großteil der Rechenzeit, der auf die Berechnung des GBuffers und der globalen Beleuchtung fällt, sind wir hingegen 
mit den Schritten Sorting und Retargeting (+ temporales Projizieren) sowohl auf  CPU als auf GPU Seite im niedrigen Prozentbereich.
Dabei wurde hier eine Blockgröße (siehe Abschnitt \ref{subsec:Blockgröße}) von B = 64 verwendet. Bei einer kleineren Blockgröße von B = 16, welche 
bereits gute Ergebnisse liefert, reduziert sich der Aufwand auf ein Viertel. 
\par
\newcolumntype{Y}{>{\raggedleft\arraybackslash}X}
\begin{figure}[H]
    \begin{tcolorbox}[title=Rechenaufwand]
        \begin{tcolorbox}[tabularx={X|Y|Y},title=Pipeline, colbacktitle=yellow!50!red, coltitle=white]
            \textbf{Stufen}                                     &  \textbf{Rechenzeit(ms/\%)} CPU & \textbf{Rechenzeit(ms/\%)} GPU \\\hline\hline
            \textbf{Gesamt}                                     &  29.55/100\%                    & 19.04/100\%\\\hline
            GBuffer                                             &  06.48/21.91\%                  & 01.30/6,83\%\\\hline
            Retargeting(+ optionale temporale Reprojektion)     &  01.12/3.8\%                    & 00.11/0.57\%\\\hline
            GGXGlobalIllumination                               &  21.20/71.74\%                  & 15.51/81,46\%\\\hline\hline
            Sorting(B=64)                                       &  00.75/2,53\%                   & 02.12/11,13\%\\\hline\hline
            \textbf{Nicht in Gesamt}                            &                                 &              \\\hline\hline
            Sorting(B=16)                                       &  00.75                          & 00.55        \\\hline\hline
        \end{tcolorbox}  
        \tcblower
        \begin{tcolorbox}[tabularx={X|Y|Y},title=Vorberechnungen, colbacktitle=yellow!50!red, coltitle=white]
            \textbf{Vorberechnung}        &  \textbf{Rechenzeit(m)}    &  \textbf{Fehler(raw/\%)}\\\hline\hline
            Retargeting                   &  17,07                     &  10714 $\approx$ 3.07\%\\\hline
            Retargeting                   &  35,36                     &  9444 $\approx$ 2,71\%\\\hline
            Temporale Projizieren         &  1382($\approx$23 h) &  $\approx$ 10000\\\hline\hline
        \end{tcolorbox}  
    \end{tcolorbox}
    \caption{Rechenzeiten die auf die einzelnen Stages fallen}
    \medskip
    \small
    * Hardware: AMD Ryzen 5 2600X, NVIDIA GeForce RTX 2060 SUPER\newline
    * Fehler: Anteil orientiert sich am Anfangsfehler von 348180 $\rightarrow$ Summe von pixelweise Differenz Blue Noise(t=0) und Blue Noise(t=1) 
\end{figure}

Um zu einem Ergebnis wie in Figur \ref{fig:annelaing animated}(siehe Abschnitt \nameref{ch:Content2:sec:Simulated Annealing}) zu kommen, braucht man $\approx17$ Minuten. 
Diese Berechnung liefert ein gutes Ergebnis bei einem akzeptablen Rechenaufwand. Für einen Fehler von 9444 (minimale Verbesserung) sind bereits $\approx35$ Minuten 
nötig.\par 
Für das temporale Projizieren benutzen wir 81 solcher modifizierten Retarget-Texturen. Dementsprechend länger braucht die Vorberechnung.\textbf{Anmerkung}: Der Prozess 
lässt sich leicht parallelisieren und damit (sehr) stark Beschleunigen. Wir haben hier davon abgesehen, da uns eine einmaliger Durchlauf genügt. \par
Die langen Ausführungszeiten auf CPU-Seite bei der GBuffer- und GGXGlobalIlluminationberechnung sind dabei mit dem Warten auf Resultate der GPU Seite zu erklären. 

%%%%%%%%%%%%%%%%%%%
%%%%% storage usage
%%%%%%%%%%%%%%%%%%%
Wir speichern unsere Anfangswerte in einer 1920x1080 32-Bit-Textur. Die \nameref{ch:Content1:sec:blue noise} Textur innerhalb des \nameref{ch:Content2:sec:Sorting} Schrittes 
hat eine Auflösung von 64x64 32-Bit. Die Retarget-Textur benutzt eine 64x64 16-Bit Textur. Das temporale Projizieren benutzt 81 solcher modifizierten Retarget-Texturen.
\begin{figure}[H]
    \begin{tcolorbox}[tabularx={X|Y},title=Speicherbedarf, colbacktitle=yellow!50!red, coltitle=white]
        \textbf{Pipelinestage}  &  \textbf{Speicherbedarf/KB} \\\hline\hline
        Sorting                 &  16,1                     \\\hline
        Retargeting             &  12,1                    \\\hline
        Temporale Reprojektion  &  980                    \\\hline
        Anfangswerte(Seeds)     &  7920                     \\\hline\hline
        \textbf{Gesamt}         &  \textbf{8928,2}           \\\hline\hline                
    \end{tcolorbox}
    \caption{Speicherdarf}
\end{figure}

Die Texturen für das Retargeting sowie für das temporale Projizieren wurden unkomprimiert verwendet. Dabei würde dies ein weiteres hohes
Speichereinsparpotenzial bieten. Bei den 8 Bit pro Farbkanal stehen uns 256 Werte zum Abspeichern bereit. Dabei benutzen wir im Retarget-Schritt 
nur sieben und dem Projizieren nur zehn.
    