Während unseren Arbeiten zu \nameref{ch:Content1:sec:blue noise} Fehlerverteilungen im Bildraum ergaben sich
weitergehende, lohnenswerte Untersuchungen, die jedoch den Umfang dieser Arbeit überstiegen hätten.

\begin{figure}[H]
    \begin{tcolorbox}[tabularx={X|X},title=\textbf{Zukünftiges}, colbacktitle=green!50, coltitle=black]
        \textbf{Untersuchungen}                         &  \textbf{Beschreibung} \\\hline\hline
        Temporales Projizieren und höhere
        Dimensionalität des \nameref{ch:Content1:sec:Path Tracer}       &  Wir haben den hier eingeführten Schritt des
                                                                        temporalen Projizierens anhand niederdimensionalen
                                                                        Rendering Integralen getestet.
                                                                        Die Funktionalität mit höheren Dimensionen würde den
                                                                        Schritten des \nameref{ch:Content2:sec:Sorting} und 
                                                                        \nameref{ch:Content2:sec:Retargeting} entsprechen.\\\hline
        Ausbauen GUI bei Retargeting-Textur Berechnung                  &  Die Berechnungen der Retargeting/Projektion-Texturen sowie der 
                                                                        Visualisierungen lassen sich bereits simultan in separaten Threads
                                                                        mit benutzerspezifischen Eingaben berechnen. Allerdings sind die 
                                                                        für den Nutzer wählbaren Parameter beschränkt. Ein weiterer Ausbau 
                                                                        steigert die Wiederverwertbarkeit und vereinfacht andere
                                                                        zukünftige Arbeiten.\\\hline 
        Parallelisieren der Texturberechnung temporales Projizieren     &  Der Vorgang lässt sich leicht parallelisieren und man könnte ihn dementsprechend
                                                                        stark beschleunigen. Wir haben hier erstmal davon abgesehen, da es eine einmalige
                                                                        Berechnung ist.                                                                         
    \end{tcolorbox}
    \label{pic:FutureWork}
\end{figure}