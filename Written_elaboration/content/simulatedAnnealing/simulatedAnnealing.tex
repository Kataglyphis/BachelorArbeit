Für unseren temporalen Algorithmus (siehe Abschnitt \ref{ch:Temporaler Algorithmus}) gibt es einen wichtigen \nameref{ch:Content2:sec:Retargeting} Schritt.
In diesem Schritt wird eine vorberechnete Textur verwendet. Diese speichert eine Permutation, die unsere \nameref{ch:Content1:sec:blue noise} Textur vom Bild t in eine
\nameref{ch:Content1:sec:blue noise} Textur von Bild t+1 umwandelt. Diese Permutation wird dann auf die Startwerte angewandt bevor das nächste Bild t+1 erzeugt wird 
(siehe auch Übersicht \ref{pic:Render Graph}). Dadurch werden die \nameref{ch:Content1:sec:blue noise} Umverteilungen der \nameref{ch:Content2:sec:Sorting} Phase akkumuliert. 
All diese Vorberechnungen sind möglich, da wir mit \textit{\glqq nur\grqq} quasi-zufälligen Sequenzen (siehe Abschnitt \ref{ch:Content1:sec:Quasi-Zufallsfolgen}) arbeiten.
Das andauernde Permutieren von Pixeln bis zu einem Punkt, an dem ein Bild aussieht wie das Andere ist ein klassisches TSP, wofür es aktuell keine effiziente optimale Lösungsmethode gibt.
Da wir nur an einer sehr guten Lösung, nahe dem globalen Optimum, interessiert sind greifen wir wie in \cite{hal02158423} vorgeschlagen auf das heuristische Approximationsverfahren,
dem Simulated Annealing, zu.
Aus der Metallurgie kommend fand das Simulated Annealing seine Verwendung in die Algorithmik. So fand man ursprünglich bei erhitztem Metall heraus, dass man durch einen 
kontrollierten Abkühlprozess den einzelnen Atomen ausreichend Zeit geben kann eine feste Ordnung und damit feste Kristalle zu bilden. Diese Ordnung entspricht der 
physikalischen Vorstellung eines energetisch günstigsten Zustandes.

\subsubsection{Algorithmik}

Angelehnt an die Erreichung eines energetisch günstigsten Zustandes definieren wir zuerst die Energiefunktion für unsere Textur als pixelweisen Unterschied
der sich in Abkühlung 
befindlichen bereits permutierten Textur und der $Textur_{t+1}$ (siehe Abbildung \ref{eq:pixel energy function}). $Textur_{t+1}$ ist durch quasi Zufall (siehe Abschnitt
\ref{ch:Content1:sec:Quasi-Zufallsfolgen}) bereits bekannt. Mit Erkenntnissen aus\cite{georgiev2016blue} ergibt sich

\begin{figure}[H]
    \begin{tcolorbox}[rightrule=3mm, rounded corners=east]
    \[ E(SA) = \sum_{p \neq q}E(p,q) = \sum_{\forall i \in [0,N-1]} \Vert{p_{i}-q_{i}}\Vert \]
    \end{tcolorbox}
    \caption{\nameref{ch:Content1:sec:blue noise} Textur mit Dimension N; Pixel p von abkühlende 
    $Textur_{t=0}$; Pixel q von $Textur_{t=1}$}
    \label{eq:pixel energy function}
\end{figure}

Da wir in jedem Schritt nur eine Permutation anwenden, vereinfacht sich unsere Energiefunktion
\ref{eq:pixel energy function} zu 

\begin{figure}[H]
    \begin{tcolorbox}[rightrule=3mm, rounded corners=east]
    \[ E(SA) = E(s_{previous}) + \Vert{p_{i}-q_{i}}\Vert + \Vert{p_{i 
        + permutation}-q_{i + permutation}}\Vert\]
    \end{tcolorbox}
  \caption{Zustand $s_{previous}$ ohne angewandte Permutation}
  \label{eq:vereinfachte pixel energy function}
\end{figure}

\newpage
Mit dem definierten Ziel, die Energiefunktion \ref{eq:pixel energy function} zu minimieren und damit einen energieärmeren Zustand anzunehmen,
wenden wir in jedem Schritt eine Permutation an. Das physikalische Vorbild, das metallurgische Abkühlen, orientiert sich an 
der Boltzmann-Statistik \ref{eq:Boltzmann-Statistik}.

\begin{figure}[H]
    \begin{tcolorbox}[rightrule=3mm, rounded corners=east]
    \[ \exp(-\frac{E_{j}}{k_{B}*T}) \]
    \end{tcolorbox}
    \caption{Boltzmann-Statistik}
    \label{eq:Boltzmann-Statistik}
    \medskip
    \small
    * $k_{B}$ Boltzmann-Konstante: Hat Dimension Energie/Temperatur; verbindet als Proportionalitätsfaktor
    Zustände mit ihrer Entropie
\end{figure}

Dadurch lässt sich physikalisch die Wahrscheinlichkeit einen Energiezustand $E_{j}$ anzutreffen formulieren 
($E_{j}\propto Gleichung$ \ref{eq:Boltzmann-Statistik}) \cite{Kirkpatrick671}. Mit diesem physikalischen Vorbild untersuchen 
wir unsere Akzeptanzwahrscheinlichkeitsfunktion für einen neuen Energiezustand: 

\begin{figure}[H]
    \centering
    \begin{subfigure}[b]{0.4\textwidth}
        \begin{tcolorbox}[rightrule=3mm, rounded corners=east]
            \begin{equation}\label{eq:Akzeptanzwahrscheinlichkeitsfunktion}
                P = \exp(-\frac{\Delta_{energy}}{temperature})
            \end{equation}
        \end{tcolorbox}
        \caption{Akzeptanzwahrscheinlichkeitsfunktion; Abhängig von Energie und aktueller Temperatur}
    \end{subfigure}
    \begin{subfigure}[b]{0.7\textwidth}
        \centering \includegraphics[interpolate=false,width=\linewidth]{content/simulatedAnnealing/Bilder/exponentialfunktion_as_PDF.png}
        \caption{Günstige Eigenschaft der Exponentialfunktion}
        \label{fig:Exponentialfunktion}
    \end{subfigure}
    \caption{}
\end{figure}

Die günstigen Eigenschaften der Exponentialfunktion (siehe Abbildung \ref{fig:Exponentialfunktion}) als Wahrscheinlichkeitsakzeptanzfunktion sind vielfältig und bereits in weiterführender Literatur wie 
\cite{Kirkpatrick671},\cite{van1987simulated} gut belegt. Eine der Eigenschaften ist in der obigen Abbildung \ref{fig:Exponentialfunktion} dargestellt. Das Argument der Funktion 
\ref{fig:Exponentialfunktion} hat einen Divisor Temperatur und einen Dividenten $\Delta_{energie}$. Mit absteigender Temperatur erkennt man 
in der Abbildung \ref{fig:Exponentialfunktion} eine ebenfalls abnehmende Wahrscheinlichkeit der Akzeptanz. Dies führt zu dem gewünschten Verhalten, energiehöhere Zustände 
zuzulassen, um somit lokale Maxima zu verlassen. Dies geschieht bei höheren Temperaturen häufiger wohingegen bei niederen Temperaturen ein gefundenes Maxima
seltener verlassen wird. Höhere Deltas führen passender Weise zu einem höheren negativen Exponenten und damit eine geringere Akzeptanz als Energiezustände, die nur bisschen 
drüber liegen. Die Wahl des Abkühlvorgangs (also das Updaten der Temperatur über die Zeit) ist problemspezifisch \cite[S. 9]{Kirkpatrick671}. Dabei muss der Abkühlvorgang derart
gewählt werden, sodass kein bloßer Greedy-Algorithmus entsteht und man in einem lokalen Maxima stecken bleibt aber auch kein wahlloses Vertauschen entsteht.
Diese Vorgänge habe ich im Abschnitt \ref{subsec:Abkühlfunktion} untersucht.\par
Wir stellen fest, dass sich durch die physikalische Analogie der Akzeptanzwahrscheinlichkeitsfunktion \ref{eq:Akzeptanzwahrscheinlichkeitsfunktion} zur 
Boltzmann-Statistik \ref{eq:Boltzmann-Statistik} Erkenntnisse des zweiten Hauptsatz der Thermodynamik übertragen. Dieser besagt vereinfacht, 
dass mit Energiezunahme eine Zunahme an möglichen Zustandsübergängen einhergeht.

Nun haben wir alle Begriffe zusammen um einen ersten Blick auf den Algorithmus zu werfen.
\begin{tcolorbox}
\begin{algorithm}[H]
    \caption{\textbf{Simulated Annealing}}
    \begin{algorithmic}[1]
        \State initialisiere Startzustand $s=s_{0}$
        \State initialisiere Starttemperatur $T_0$
        \For{i=1...maxSteps}
        \State update Temperatur $t_i$ anhand des Abkühlplans
        \State //Radius für Nachbarschaftssuche ist auf 6 festgesetzt
        \State $s_{neu}\leftarrow$Nachbarzustand(s) //wende hier die Permutation an!
        \State $energy\Delta = energy(s_{neu}) - energy(s)$
        \If{$energy\Delta < 0$}
        \State s = $s_{new}$
        \Else{}
        \If{P(Energie(s), Energie($s_{new}$), temperature)$\ge$ random(0,1)} 
        \State s = $s_{new}$
        \EndIf
        \EndIf
        \EndFor
        \State return Endzustand s;
    \end{algorithmic}
    \label{alg:retargeting}
\end{algorithm}
\end{tcolorbox}

Für die abzuspeichernde Permutation gilt Folgendes. Als Startzustand $s_{0}$ definieren wir eine Permutation, die alle Elemente auf sich selbst abbildet.
Um von einem Zustand s zu einem neuem Zustand $s_{new}$ zu kommen, definieren wir eine Nachbarschaftsfunktion \textit{Nachbarzustand()}. 
Diese kann zwei Elemente genau dann vertauschen, wenn Sie in einem gegenseitigen Radius r = 6 erreichbar sind (folgend der Empfehlung aus \cite[S.7]{hal02158423})
Dabei vertauschen wir in jedem Schritt ein Pixelpaar. Die Wahrscheinlichkeitsfunktion zur neuen Zustandsannahme P(Energie(s), Energie($s_{new}$)) beschreibt, ob wir den neu
gewählten Zustand $s_{new}$ übernehmen. Dabei wird klassischerweise die Akzeptanz von Zuständen mit höherer Energie immer kleiner.(bzw. die 
Toleranz gegenüber größeren Fehlern im Bezug zur Zeit). Die allgemeine Akzeptanz von Zuständen mit höherer Energie ist dabei von fundamentaler Bedeutung.
Somit verlassen wir möglicherweiße nur lokale Maxima.
\newpage

\subsection{Abkühlfunktion}
\label{subsec:Abkühlfunktion}

Für die Wahl unserer Abkühlfunktion bieten sich einige Möglichkeiten (siehe Abbildung \ref{pic:Cool Down Comparisson}). Im Folgenden wird auf die verschiedenen möglichen Abkühlfunktionen 
(\cite{ScienceDirectCoolingSchedule}) und ihre Eigenschaften sowie die Wahl interner Parameter(z.B. Starttemperatur, Gleichgewicht) eingegangen. Denn diese Funktion
trägt maßgeblich mit ihrem Konvergenzverhalten zur Effizienz des Abkühlvorgangs bei. So beeinflusst Sie auch unsere wichtige Wahrscheinlichkeitsakzeptanzfunktion \ref{eq:Akzeptanzwahrscheinlichkeitsfunktion}.
Nach \cite{Kirkpatrick671} wählen wir die Anfangstemperatur $T_0$ derart, dass anfangs jede Neue generierte Lösung akzeptiert wird. Außerdem werden wir einen Zustand des Quasiequilibriums (Gleichgewicht) definieren.
Für einige Abkühlvorgänge wird es sinnvoll sein, erst nach einer bestimmten Anzahl von erfolgreichen Zustandsübergängen die Temperatur zu senken. Dazu beim jeweiligen Vorgang mehr.

\textbf{Hajek}
\begin{tcolorbox}[rightrule=3mm, rounded corners=east]
    \begin{equation}\label{eq:Hajek}
        f(t) = T_0\log(1+t)
    \end{equation}
\end{tcolorbox}

In \cite{hajek1988cooling} haben wir eine Abkühlfunktion gegeben, welche durch ihre Eigenschaft,
stets gegen das globale Maximum zu konvergieren, eine Funktion die unter allen Anderen heraussticht.
In Abbildung \ref{pic:Cool Down Comparisson} angedeutet und in weiteren Beobachtungen bestätigt hat sich 
allerdings auch ihre sehr langsame Konvergenz.
Sie hat sich daher für diese Aufgabenstellung als nicht nützliche Abkühlfunktion herausgestellt.

\textbf{Linear}
\begin{tcolorbox}[rightrule=3mm, rounded corners=east]
    \begin{equation}\label{eq:lineare Abkühlung}
        f(t) = T_0 - \mu*t
    \end{equation}
\end{tcolorbox}

Typische Werte für $\alpha$ liegen zwischen 0.8 and 0.99. Wie man in Abbildung \ref{pic:Cool Down Comparisson}
erkennen kann, ist das Problem der linearen Abkühlung die extreme Langsamkeit.
Anstatt nur am Anfang schlechtere Energiezustände zuzulassen um lokale Minima zu verlassen 
geht der Algorithmus durch diesen Abkühlvorgang in ein bloßes randomisiertes Tauschen von Pixeln über.

\textbf{Exponential}

\begin{tcolorbox}[rightrule=3mm, rounded corners=east]
    \begin{equation}\label{eq:Exponential}
        f(t) = T_0*pow(\alpha,t)
    \end{equation}
\end{tcolorbox}

Ist nach \cite{Kirkpatrick671} eine für viele Fälle zutreffende und zu wählende Abkühlfunktion.
Wobei $\alpha \in [0.8; 0.99]$.


Wie in Abbildung \ref{pic:Cool Down Comparisson} zu erkennen haben wir hier hingegen zu den vorherigen Vorgängen
eine deutlich schnellere Abkühlung. Jedoch lässt sich hier das andere Extrem, im Vergleich zum bloßen randomisierten
Vertauschen von Pixelpaaren, erkennen: Wir verharren viel zu kurz in einem Temperaturzustand, geraten daher schnell 
in einen \textit{greedy} Zustand und manche Bildbereiche bleiben in einem lokalen Minima hängen.

\textbf{Inverse}

\begin{tcolorbox}[rightrule=3mm, rounded corners=east]
    \begin{equation}\label{eq:Inverse}
        f(t) = T_0 / (1 + alpha * step)
    \end{equation}
\end{tcolorbox}

Wie in Abbildung \ref{pic:Cool Down Comparisson} zu erkennen haben wir hier hingegen zu ersten beiden Vorgängen
eine deutlich schnellere Abkühlung. Hat jedoch das selbe Problem wie die exponentielle \ref{eq:Exponential} Variante.

\begin{figure}[H]
    \centering
    \includegraphics[width=\linewidth]{content/simulatedAnnealing/Bilder/Energy_Cooldown_compared_steps_85771.png}
    \caption{Vergleich von Abkühlfunktionen mit gesetzten Parametern
            alle mit 100000 Schritten; auf x-Achse sind erfolgreiche Schritte nach Wahrscheinlichkeitsakzeptanzfunktion
            aufgetragen}
    \label{pic:Cool Down Comparisson}
\end{figure}

%%%%%%%%%%%%%%%%%%%%%%%%%%%%%%%%%%%%%%%%%%%%%%%%%%%%%%%%%%%%%%%%%%%%%%%%%%%%%%%%%%%%%%%%%%%%%%%%%%%%%
%%%%%%%%%%%%%%%%%%%%%%% this is our cooling schedule we are going for %%%%%%%%%%%%%%%%%%%%%%%%%%%%%%%
%%%%%%%%%%%%%%%%%%%%%%%%%%%%%%%%%%%%%%%%%%%%%%%%%%%%%%%%%%%%%%%%%%%%%%%%%%%%%%%%%%%%%%%%%%%%%%%%%%%%%

\textbf{Kirkpatrick}
Wir haben uns bei unserem Optimierungsproblem für einen Abkühlvorgang, der in \cite{Kirkpatrick671} beschrieben wird, 
entschieden und in danach benannt. Die Abkühlfunktion hat hierbei exponentiellen Charakter. 

\begin{tcolorbox}[rightrule=3mm, rounded corners=east]
    \begin{equation}\label{eq:ExponentialKirkpatrick}
        f(t) = T_0 * pow(\alpha,t)
      \end{equation}
\end{tcolorbox}

Wobei wieder $\alpha \in [0.8; 0.99]$ und nach Auflösung der Textur zu wählen ist. Ein anderer Parameter, der nach 
Auflösung der Textur zu wählen ist, ist das Quasi-Gleichgewicht. Mit dem Quasi-Gleichgewicht lässt sich erreichen, 
dass jeder Bildabschnitt vor dem jeweiligen Abkühlen jede Temperatur durchläuft und dabei jeden Bildabschnitt davor 
bewahrt in einen lokalen Minima zu verharren.
So lässt sich in einem direkten Vergleich mit dem bloßen exponentiellen Abkühlen in Abbildung 
\ref{pic:Cool Down Comparisson} erkennen, das wir anfangs langsamer abkühlen und immer wieder auch 
höhere Energiezustände bewusst zu lassen.
\vfill

\subsection{Visualisierung}

\begin{figure}[H]
    \centering
    \includegraphics[width=0.8\linewidth]{content/simulatedAnnealing/Bilder/Energy_286100_steps_KirkpatrickCooldownSchedule.png}
    \caption{Energieverlauf beim Simulated Annealing}
    \label{pic:kirkpatrick energie verlauf}
\end{figure}

Die Abbildung \ref{pic:kirkpatrick energie verlauf} verdeutlicht, dass wir durch unser Abkühlen insgesamt die 
Energiefunktion \ref{eq:pixel energy function} minimieren. Dabei akzeptieren wir in Abhängigkeit unserer Schritte,
anfangs sehr häufig und am Ende immer seltener, neben günstigeren Zuständen, auch energetische Höherwertigere.

\begin{figure}[H]
    \centering
    \includegraphics[width=0.8\linewidth]{content/simulatedAnnealing/Bilder/Temperature.png}
    \caption{Temperaturverlauf}
    \label{pic:Temperaturverlauf kirkpatrick}
\end{figure}

Wir wählen die Starttemperatur (siehe Abbildung \ref{pic:Temperaturverlauf kirkpatrick}) folgendermaßen, dass 
anfangs alle Permutationen akzeptiert werden. Dazu muss die 
Akzeptanzwahrscheinlichkeitsfunktion \ref{eq:Akzeptanzwahrscheinlichkeitsfunktion} auch die energetisch
ungünstigste Permutation akzeptieren. Minimiere Argument der Exponentialfunktion \ref{fig:Exponentialfunktion}.
In unseren konkreten Fall: Maximales $\Delta_{Energy}$ bei einem 8-Bit Graustufenbild 2*255 = 510. Jeweilige 
Anpassungen müssen bei anderer Auflösung vorgenommen werden.

\begin{figure}[H]
    \centering
    \includegraphics[width=\linewidth]{content/simulatedAnnealing/Bilder/Akzeptanzwahrscheinlichkeit.png}
    \caption{Akzeptanzfunktionsverlauf bei negativen Energiedeltas}
    \label{pic:Akzeptanzfunktionsverlauf kirkpatrick}
\end{figure}

Abbildung \ref{pic:Akzeptanzfunktionsverlauf kirkpatrick} visualisiert die Akzeptanz (Dimension Prozent) im Verlauf 
des Algorithmus (Schritte) bezogen auf die mögliche Verschlechterung bzw. den wachsenden Fehler. Anfangs werden sehr viele 
und sehr große Fehler akzeptiert, wohingegen im weiteren Verlaufe vorallem im höheren Fehlerbereich der blaue Bauch auf eine 
schlechtere Akzeptanz höherer Fehler hindeutet.

\begin{figure}[H]
    \centering
    \begin{minipage}[t]{0.45\linewidth}
        \centering
        \includegraphics[interpolate=false,width=\linewidth]{content/simulatedAnnealing/Bilder/LDR_RGBA_0_64-RGBA_r_channel.png}
        \caption{Blue noise Textur 64x64}
    \end{minipage}
    \hfill
    \begin{minipage}[t]{0.45\linewidth}
        \centering
        \includegraphics[interpolate=false,width=\linewidth]{content/simulatedAnnealing/Bilder/permutation_texture_295744_swapsKirkpatrickCooldownSchedule.png}
        \caption{Permutation; gespeichert in R,G-Channel einer PNG}
        \label{pic:Retargeting textur}
    \end{minipage}
\end{figure}

%%%%%%%%%%%%%%%%%%%%%%%%%%%%%%%%%%%%%%%%%%%%%%%%%%%%%%%%%%%%%%%%%%%%%%%%%%%%%%%%%%%%%%%%%%%%%%%%%%%%%%%%%%%%%%%%%%%%%%%%%%%%%%%%%%
%%%% Animate the annealing!!!!!! %%%%%%%%%%%%%%%%%%%%%%%%%%%%%%%%%%%%%%%%%%%%%%%%%%%%%%%%%%%%%%%%%%%%%%%%%%%%%%%%%%%%%%%%%%%%%%%%%
%%%%%%%%%%%%%%%%%%%%%%%%%%%%%%%%%%%%%%%%%%%%%%%%%%%%%%%%%%%%%%%%%%%%%%%%%%%%%%%%%%%%%%%%%%%%%%%%%%%%%%%%%%%%%%%%%%%%%%%%%%%%%%%%%%

\begin{figure}[H]
    \begin{tcolorbox}[boxrule=4pt,sharp corners=downhill,title=Abkühlen]
    \centering
    \begin{subfigure}[b]{0.2\linewidth}
      \includegraphics[width=\linewidth]{content/simulatedAnnealing/Bilder/Annealing/LDR_RGBA_0_64-RGBA_r_channel.png}
       \caption{\nameref{ch:Content1:sec:blue noise} t}
       \label{pic:dither0}
    \end{subfigure}
    \begin{subfigure}[b]{0.2\linewidth}
      \includegraphics[width=\linewidth]{content/simulatedAnnealing/Bilder/Annealing/intermediate_applied_permutation_0_quasieqstepKirkpatrickCooldownSchedule_energy_348180-RGBA_r_channel.png}
      \caption{348180}
      \label{pic:abkühl_schritt_1}
    \end{subfigure}
    \begin{subfigure}[b]{0.2\linewidth}
      \includegraphics[width=\linewidth]{content/simulatedAnnealing/Bilder/Annealing/intermediate_applied_permutation_1_quasieqstepKirkpatrickCooldownSchedule_energy_275698-RGBA_r_channel.png}
      \caption{275698}
      \label{pic:abkühl_schritt_2}
    \end{subfigure}
    \begin{subfigure}[b]{0.2\linewidth}
        \includegraphics[width=\linewidth]{content/simulatedAnnealing/Bilder/Annealing/intermediate_applied_permutation_2_quasieqstepKirkpatrickCooldownSchedule_energy_174766-RGBA_r_channel.png}
         \caption{174766}
         \label{pic:abkühl_schritt_3}
      \end{subfigure}

    %%%%%%%%%%%%%%%%%%%%%%%%%%%%%%%%%%%%%%%%%%%%%%%%%%%%%%%%%%%%%%%%%%%%%%%%%%%%%%%%%%%%%%%%%%%%%%%%%%%%%%
    %%%%%%%%%%%%%%%%%%%%%%%%%%%%%%% second row
    %%%%%%%%%%%%%%%%%%%%%%%%%%%%%%%%%%%%%%%%%%%%%%%%%%%%%%%%%%%%%%%%%%%%%%%%%%%%%%%%%%%%%%%%%%%%%%%%%%%%%%

    \begin{subfigure}[b]{0.2\linewidth}
        \includegraphics[width=\linewidth]{content/simulatedAnnealing/Bilder/Annealing/intermediate_applied_permutation_3_quasieqstepKirkpatrickCooldownSchedule_energy_86398-RGBA_r_channel.png}
        \caption{86398}
        \label{pic:abkühl_schritt_4}
    \end{subfigure}
    \begin{subfigure}[b]{0.2\linewidth}
        \includegraphics[width=\linewidth]{content/simulatedAnnealing/Bilder/Annealing/intermediate_applied_permutation_4_quasieqstepKirkpatrickCooldownSchedule_energy_52550-RGBA_r_channel.png}
        \caption{52550}
        \label{pic:abkühl_schritt_5}
    \end{subfigure}
    \begin{subfigure}[b]{0.2\linewidth}
        \includegraphics[width=\linewidth]{content/simulatedAnnealing/Bilder/Annealing/intermediate_applied_permutation_8_quasieqstepKirkpatrickCooldownSchedule_energy_19742-RGBA_r_channel.png}
         \caption{19742}
         \label{pic:abkühl_schritt_6}
    \end{subfigure}
    \begin{subfigure}[b]{0.2\linewidth}
        \includegraphics[width=\linewidth]{content/simulatedAnnealing/Bilder/Annealing/intermediate_applied_permutation_15_quasieqstepKirkpatrickCooldownSchedule_energy_12056-RGBA_r_channel.png}
        \caption{12056}
        \label{pic:abkühl_schritt_7}
    \end{subfigure}

    %%%%%%%%%%%%%%%%%%%%%%%%%%%%%%%%%%%%%%%%%%%%%%%%%%%%%%%%%%%%%%%%%%%%%%%%%%%%%%%%%%%%%%%%%%%%%%%%%%%%%%
    %%%%%%%%%%%%%%%%%%%%%%%%%%%%%%% third row
    %%%%%%%%%%%%%%%%%%%%%%%%%%%%%%%%%%%%%%%%%%%%%%%%%%%%%%%%%%%%%%%%%%%%%%%%%%%%%%%%%%%%%%%%%%%%%%%%%%%%%%

    \begin{subfigure}[b]{0.2\linewidth}
        \includegraphics[width=\linewidth]{content/simulatedAnnealing/Bilder/Annealing/intermediate_applied_permutation_19_quasieqstepKirkpatrickCooldownSchedule_energy_10714-RGBA_r_channel.png}
        \caption{10714}
        \label{pic:abkühl_schritt_8}
    \end{subfigure}
    \begin{subfigure}[b]{0.2\linewidth}
        \begin{picture}(120,120)
            \put(25,50){\Huge $\rightarrow$}
        \end{picture}
    \end{subfigure}
    \begin{subfigure}[b]{0.2\linewidth}
        \includegraphics[width=\linewidth]{content/simulatedAnnealing/Bilder/Annealing/difference.png}
        \caption{Differenz}
        \label{pic:dither1}
    \end{subfigure}
    \end{tcolorbox}
    \caption{Der Prozess des Abkühlens mit jeweiliger Auswertung der Energiefunktion \ref{eq:pixel energy function}}
    \label{fig:annelaing animated}
\end{figure}

  Nachdem wir in Bild 
  \begin{itemize}
    \item[\ref{pic:dither0}] unsere \nameref{ch:Content1:sec:blue noise} Textur zum Zeitpunkt t=0 
                              haben können wir in dem Bild
    \item[\ref{pic:abkühl_schritt_2}] typische Clusterbildungen erkennen, die die Folge von 
                                    uniformen randomisierten Vertauschungen sind. Diese hatte wir bereits 
                                    im Abschnitt \ref{ch:Content1:sec:blue noise} und \ref{ch:Content1:sec:Quasi-Zufallsfolgen}
                                    besprochen. Diese Vertauschungen sind die Folge der hohen Temperatur, welche wiederrum zur Folge haben, 
                                    dass alle Vertauschungen von der Funktion \ref{eq:Akzeptanzwahrscheinlichkeitsfunktion} angenommen werden. 
                                    Die Bilder  
    \item[\ref{pic:abkühl_schritt_2}-\ref{pic:abkühl_schritt_4}] sind die Folge der abnehmenden Temperatur und der daraus folgenden geringeren 
                                    Akzeptanz schlechterer, energiereicherer Zustände. Bessere Zustände werden allerdings immer angenommen und 
                                    daher die immer bessere \nameref{ch:Content1:sec:blue noise} Verteilung. Die bessere Verteilung ist eben über 
                                    die gesamte Textur zu erkennen. Mit abnehmender Energie in den Bildern   
    \item[\ref{pic:abkühl_schritt_6}-\ref{pic:abkühl_schritt_8}] lassen wir sehr wenige/keine schlechteren Zustände mehr zu und gelangen durch 
                                    lokale Permutationen zu einer immer exakteren Verteilung.  
    \item[\ref{pic:dither1}] Pixelweise Differenz mit Zieltextur; Durchschnittlicher Fehler/Pixel von $\approx 2,6$ 
  \end{itemize}





